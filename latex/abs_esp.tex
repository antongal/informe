\begin{abstract}

Cuando se trata de validar la correcci'on de sistemas inform'aticos, especialmente en desarrollos cr'iticos, el \emph{testing} no es suficiente y es necesario recurrir a t'ecnicas de an'alisis formal y herramientas con fundamentos matem'aticos s'olidos. 

En estos casos, es casi indispensable utilizar lenguajes con alto poder expresivo y herramientas de an'alisis que provean respuestas completas.
Algunos demostradores de teoremas entran en esta categor'ia.
Por su parte, herramientas tales como buscadores de contraejemplos o \emph{model checkers} por lo general no son adecuados para sistemas cr'iticos, debido a que s'olo realizan an\'alisis parciales.
Como contracara, estos 'ultimos funcionan autom'aticamente, mientras los demostradores de teoremas m'as poderosos necesitan la gu'ia de una persona para completar el an'alisis y usualmente tienen una curva de aprendizaje muy empinada.

Transversalmente a este inconveniente, los desarrollos actuales son esencialmente heterog'eneos. Programas ejecut'andose en distintas plataformas (tablets, tel'efonos celulares, servidores) trabajan cooperativamente a diario, comunic'andose a trav'es de la Internet. En la mayor'ia de los casos, estos programas son desarrollados no s'olo por distintas personas sino tambi'en utilizando metodolog'ias y lenguajes de programaci'on diferentes.

%A su vez los demostradores de teoremas presentan varios problemas como por ejemplo su dificultad de uso y algunas limitaciones, en particular introducidas por la expresividad del lenguaje en el que se presenta el an'alisis.

Heterogenius \cite{heterogenius} es un herramienta de an'alisis heterog'eneo % demostrador de teoremas heterogeneo e interactivo. 
cuyo objetivo es atacar estas limitaciones al presentar un entorno de an'alisis donde se pueden combinar m'etodos autom'aticos e interactivos y utilizar distintos lenguajes durante un mismo proceso de an'alisis.

%Adicionalmente, provee una interfaz gr'afica, representar la demostraci'on en forma de un 'arbol, denominado \emph{'arbol de an'alisis}, y permitir el uso de m'ultiples lenguajes l'ogicos para especificar problemas.

%Si bien soluciona varios de los puntos marcados anteriormente, cuenta con sus propios inconvenientes. Por un lado 

% Sin embargo, como la mayor'ia de los demostradores que usan una representaci'on arb'orea, el 'arbol de an'alisis est'a limitado a mostrar un 'unico camino de demostraci'on, que es el camino que lleva al resultado exitoso y no permite documentar los intentos fallidos o mostrar alternativas. %Por otro lado, no permite tener secuentes heterogeneos con lo cual las demostraciones no son verdaderamente heterogeneas.

El trabajo de esta tesis se enfoc'o en introducir diversas mejoras a Heterogenius.
En primer lugar, se incorpor'o un nuevo tipo de an'alisis autom'atico, basado en una traducci'on a lenguajes de primer orden y en el uso de demostradores autom'aticos para dichos lenguajes.
Estas herramientas nos permiten proveer tanto demostraciones como contraejemplos en an'alisis de propiedades escritas en lenguajes con mayor poder expresivo.
Aun con esta diferencia en las expresividades de los lenguajes pudimos dotar a Heterogenius de nuevos medios que permitieron brindar resultados autom'aticos con mayor nivel de confianza que en la versi'on anterior.

Tambi'en conseguimos aumentar la granularidad de la heterogeneidad de los an'alisis. 
La hemos llevado a nivel de cada f'ormula, en lugar de una colecci'on de ellas, como comúnmente se suele hacer.
Esto permite tener mayor flexibilidad en las descripciones de una misma parte de un sistema.
Adem'as, lleva la heterogeneidad de la herramienta a un nivel nunca antes alcanzado por otras herramientas de su estilo.

En cuanto a su \emph{usabilidad}, hemos incorporado mecanismos que permiten llevar un mayor registro de las particularidades de cada análisis; permitiendo trabajar simultáneamente sobre soluciones alternativas y parciales, a diferencia de la mayoría de las herramientas tradicionales que s'olo permiten llevar adelante un intento de solución por vez.

\todomm{Agregar descripción de la evaluación empírica realizada.}

\vspace{2em}


\noindent\textbf{Palabras claves:} Ingenier'ia de software, verifici'on y validaci'on, demostraci'on de teoremas, demostraciones heterog'eneas.

\end{abstract}

%á
