\begin{abstract}

Cuando se trata de validar la correcci'on del software, especialmente en software cr'itico, el testing no es suficiente y es necesario recurrir a t'ecnicas de an'alisis formal y herramientas con fundamentos matem'aticos fuertes. En 'estos casos los demostradores de teoremas son herramientas fundamentales.

A su vez dichas herramientas presentan varios problemas como por ejemplo su dificultad de uso y algunas limitaciones, en particular introducidas por la expresividad del lenguaje en el que se presenta el an'alisis.

Heterogenius es un demostrador de teoremas heterogeneo e interactivo. Su objetivo es atacar estos problemas y limitaciones al presentar una interface gr'afica, representar la demostraci'on en forma de un 'arbol, denominado 'arbol de an'alisis y permitir el uso de m'ultiples lenguajes l'ogicos para especificar los problemas.

Si bien soluciona varios de los puntos marcados anteriormente, presenta sus propias limitaciones. Por un lado el 'arbol de an'alisis est'a limitado a mostrar un 'unico camino de demostraci'on, que es el camino que lleva al resultado exitoso y no permite documentar los intentos fallidos o mostrar alternativas. Por otro lado no permite tener secuentes heterogeneos con lo cu'al las demostraciones no son verdaderamente heterogeneas.

En 'esta t'esis solucionamos los problemas anteriormente mencionados. Primero ampliando el  'arbol de an'alisis y permitiendo las funcionalidades mencionadas. Segundo extendiendo el concepto de demostraciones heterog'eneas al permitir un mayor nivel de granularidad y permitiendo el manejo de secuentes heterogeneos. 

Por 'ultimo, debido a la gran oferta de herramientas autom'aticas de l'ogica de primer 'orden, integramos con Heterogenius algunas de las m'as usadas mediante la integraci'on del lenguaje de l'ogica de primer 'orden \textit{TPTP-FOF} y la implementaci'on de una traducci'on-$\rho$ desde el lenguaje de f'ormulas \textit{PDOCFA}.

\vspace{2em}


\noindent\textbf{Palabras claves:} Ingenier'ia de software, verifici'on y validaci'on, demostraci'on de teoremas, demostraciones heterog'eneas, tptp-fof.

\end{abstract}

%á