\chapter{Conclusiones y Trabajos Futuros}

En esta tesis se logr'o ampliar algunos conceptos tanto en la teor'ia como en la pr'actica, implementando las mejoras en \textit{Heterogenius}. Primero logramos extender el concepto del 'arbol de an'alisis, implementando ramificaciones alternativas. Esto entre otras cosas nos permiti'o crear caminos alternativos en una demostraci'on para analizar diferentes formas de resolver el problema y adem'as mantener el historial de todo el an'alisis en el 'arbol al maneter las ramas que no llevan a un resultado exitoso.

Tambi'en se ampli'o el concepto de heterogeneidad al permitir soportar secuentes heterogeneos. De esta forma el an'alisis se volvi'o m'as sencillo al introducir un mayor nivel de abstracci'on sobre los lenguajes aceptados por el sistema. Adem'as se incorporaron herramientas y acciones para traducir, seleccionar o combinar las f'ormulas dentro de un secuente dado.

Luego se extendi'o la funcionalidad de \textit{Heterogenius} con el lenguaje de l'ogica de primer 'orden \textit{TPTP-FOF} y algunas herramientas autom'aticas que trabajan con dicho lenguaje. En particular se integraron dos demostradores autom'aticos de teoremas: \textit{EProver} y \textit{SPASS} y dos buscadores de modelos finitos: \textit{EProver} y \textit{Mace4}. Para que esta integraci'on fuera posible fue necesario definir una traducci'on desde el lenguaje \textit{PDOCFA} a \textit{TPTP-FOF}, tomando en cuenta las diferencias de expresividad entre los dos lenguajes.

Por 'ultimo se evaluaron las herramientas autom'aticas integradas sobre un conjunto de lemas de XXX, mostrando cuales son los casos donde 'estas herramientas logran mejores o peores resultados.

Esta tesis abre varias posibilidades para la extensi'on de \textit{Heterogenius} con m'as herramientas que soporten el lenguaje \textit{TPTP-FOF}. Tambi'en al haber realizado un \textit{refactoreo} del c'odigo se brinda un mayor soporte de extensi'on para nuevos lenguajes como por ejemplo \textit{TPTP-HOL} un lenguaje de l'ogica de alto orden, lenguajes que trabajen sobre l'ogicas temporales, etc.

Con lo cual algunas de las mejoras y aportes que se pueden realizar son:

\begin{itemize}

\item Mejorar la interfaz y la interacci'on con el usuario, permitiendo una mayor facilidad de uso de la herramienta.

\item Extender \textit{Heterogenius} con lenguajes y herramientas autom'aticas nuevas.

\item Implementar las acciones de c'alculo de secuentes para el lenguaje \textit{TPTP-FOF} para permitir analisis y demostraciones completas en 'este lenguaje.

\item Agregar herramientas autom'aticas que trabajen con las l'ogicas relacionales para solucionar los problemas mostrados en \ref{analisis_lemas}.

\end{itemize}



%á