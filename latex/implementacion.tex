\section{Detalles de Implementaci'on}

En 'esta secci'on se explicar'an algunas de las dificultades y detalles de las decisiones tomadas al implementar los nuevos conceptos en \textit{Heterogenius}. En particular vamos a explicar la extensi'on de las traducciones $\rho$, la integraci'on del lenguaje \textit{TPTP-FOF} e implementaci'on de buscadores de contraejemplo y demostradores usando las herramientas autom'aticas que trabajan con el lenguaje \textit{TPTP-FOF}.

\subsection{Extensi'on de las traducciones Rho}

\todomm{El punto fuerte de tu trabajo fue la implementación de estos conceptos, así que tenemos que explicar bien esto como para que los jurados lo entiendan así. En principio, deberías explicar un poco cómo estaba diseñado Heterogenius, para que se entienda la magnitud del cambio que realizaste}

Debido a los cambios introducidos fue necesario \textit{refactorizar} el diseño de la infraestructura de las traducciones $\rho$. Lo primero que se hizo fue agregar un \textit{TranslationsManager}, un objeto encargado de manejar todas las traducciones soportadas por el sistema. 

Por otro lado los traductores (subclases de \textit{Translator}) deben implementar los tres m'etodos  abstractos definidos en la clase padre. Cada uno de 'estos m'etodos permite un control m'as fino de las traducciones al separar el secuente en sus partes, que son: una referencia de skolemizac'ion, una especificaci'on y la f'ormula analizada.

\begin{figure}[H]
	\includegraphics[width=400px, angle=90]{img/arq_traductores.png}
\end{figure}

Se proveen los traductores de \textit{Alloy} a \textit{PDOCFA}, de \textit{PDOCFA} a \textit{TPTP-FOF} asi como el \textit{CompoundTranslator} que permite componer los traductores para lograr traducciones transitivas, por ejemplo de \textit{Alloy} a \textit{TPTP-FOF}.


\subsection{Demostradores de teoremas}

\subsubsection{Preparaci'on del secuente}

El formato \textit{TPTP-FOF} que usan las herramientas autom'aticas agregadas es una lista de f'ormulas escritas en el lenguaje \textit{TPTP-FOF}. Cada una de 'estas f'ormulas debe tener un tipo que puede ser: \textit{axiom} o \textit{conjecture} (existen m'as tipos pero son irrelevantes para nuestro caso).

Para convertir el secuente analizado al formato \textit{TPTP-FOF} se agregan todas las f'ormulas del antecedente con el tipo \textit{axiom} y las f'ormulas del consecuente con el tipo \textit{conjecture}. 

Al ejecutar un demostrador autom'atico con la entrada preparada de este modo, se va a tratar de probar las f'ormulas del consecuente usando que las f'ormulas del antecedente son verdaderas.


\subsubsection{Integraci'on con Heterogenius}

Para integrar las herramientas autom'aticas, tanto los demostradores de teoremas como los buscadores de contraejemplos fue necesario \textit{refactorizar} y extender el diseño de algunas partes de la arquitectura de Heterogenius. Algunos de los objetivos y criterios que se usaron durante el rediseño fueron lograr abstraer las herramientas autom'aticas usadas y hacer un diseño lo suficientemente extensible y abierto para la integraci'on con nuevas herramientas en el futuro.

\begin{figure}[]
	\includegraphics[width=450px, angle=90]{img/arq_prover.png}
	\centering
	\caption{Demostradores autom'aticos usados como calculadores de secuentes.}
	\label{fofproverdiagrama}
\end{figure}

\textit{Heterogenius} en su versi'on original defini'o la interfaz \textbf{SequentCalculator} (ver fig. \ref{ejecucionherramientaautomatica}) con un s'olo mensaje \textbf{calculate}. La idea fue que un calculador de secuentes (una herramienta autom'atica) pod'ia implementar dicha interfaz y tomando en cuenta la especificaci'on e informaci'on de eskolemizaci'on realizar un c'alculo que pod'ia producir cero o m'as secuentes nuevos.

As'i la forma m'as adecuada de integrar los demostradores autom'aticos sobre el lenguaje \textit{TPTP-FOF} fue implementar la interfaz \textbf{SequentCalculator}. Debido a que hay partes de funcionalidad que las herramientas que trabajan con \textit{TPTP-FOF} comparten, decidimos extender la jerarqu'ia de los calculadores de secuentes introduciendo la clase abstracta \textbf{FOFSequentCalculator} para representar un calculador de secuentes basado en \textit{TPTP-FOF}.

A su vez, se proveen dos implementaciones de dicha clase: \textbf{FOFSequentCalculator\_overEProver} y \textbf{FOFSequentCalculator\_overSPASS} que como sus nombres lo indican son calculadores de secuentes que utilizan \textit{EProver} y \textit{SPASS} respectivamente.

Vale aclarar que ninguna herramienta autom'atica incorporada provee una \textit{API} y el procesamiento se produce ejecutando el binario de la herramienta correspondiente pasandole como parametro un archivo de texto con los axiomas y la conjetura que se quiere probar como f'ormulas \textit{TPTP-FOF} serializadas. La salida \textit{STDOUT} de la ejecuci'on luego se parsea y se interpreta para producir un resultado. 'Estos pasos se pueden ver en la fig. \ref{ejecucionherramientaautomatica}.

Primero para abstraer la ejecuci'on de un herramienta autom'atica agregamos la clase abstracta \textbf{ExternalToolRunner}. Esta clase es responsable de iniciar la ejecuci'on del archivo binario de la herramienta correspondiente con los parametros apropiados y en el formato aceptable. Presentamos tres implementaciones de dicha clase, una para cada herramienta incorporada: \textbf{EProverRunner}, \textbf{SPASSRunner} y \textbf{Mace4Runner}. Notar que en 'este punto no importa si la herramienta se utiliza como un demostrador o un buscador de contraejemplos o ambos como es el caso de \textit{EProver}.

La segunda entidad que aparece en el diagrama de la fig. \ref{ejecucionherramientaautomatica} es un \textit{parser} de la salida de la herramienta. En 'este caso tambi'en se define la clase abstracta \textbf{ProverParser} que tiene dos implementaciones: \textbf{EProverProverParser} y \textbf{SPASSProverParser} y que representa a un parser que interpreta la salida de un demostrador autom'atico.

Cada herramienta produce salidas diferentes, pero en general todas contienen alguna palabra clave o un marcador que permite interpretar el resultado.

Como se ve en la fig. \ref{fofproverdiagrama} la entidad que ``orquesta'' los componentes mencioncionados es \textbf{FOFProver}. Es un objeto que se compone con un \textbf{ProverParser} y un \textbf{ExternalToolRunner} y realiza todo el proceso de ejecuci'on de la herramienta autom'atica y el posterior an'alisis del resultado.


\begin{figure}[]
	\includegraphics[width=350px]{img/ejecucion_herramienta_automatica.png}
	\centering
	\caption{Interacci'on de \textit{Heterogenius} con \textit{EProver}.}
	\label{ejecucionherramientaautomatica}
\end{figure}

El diseño presentado hace f'acil la incorporaci'on de herramientas autom'aticas nuevas que trabajen con \textit{TPTP-FOF}. Para agregar una herramienta nueva ya sea un demostrador o un buscador de contraejemplos, primero se debe subclasificar la clase abstracta \textbf{ExternalToolRunner}. Esta clase abstrae las particularidades de ejecuci'on de la herramienta especificando los parametros necesarios. Luego se extiende la clase \textbf{ProverParser} que se encarga de procesar el texto correspondiente a la salida de la herramienta. En el caso de un buscador de contraejemplos existe una clase an'aloga que se debe subclasificar que es \textbf{ModelFinderParser}.



\subsection{B'usqueda de contraejemplos}

\subsubsection{Preparaci'on del secuente}

Tanto \textit{Mace4} como \textit{E-Prover} se usan para buscar contraejemplos de los secuentes \textit{TPTP-FOF}. Como las dos herramientas son buscadores de modelos, para preparar el secuente lo que se hace es armar una lista de f'ormulas de tipo \textit{axiom} a partir de las f'ormulas del antecedente y del consecuente del secuente analizado.

Cuando todas las f'ormulas que se env'ian como entrada a los buscadores de modelos son de tipo \textit{axiom} las herramientas tratan de buscar un modelo para el conjunto de los axiomas recibidos.

Como lo que se quiere lograr es encontrar un contraejemplo, se hace necesario negar el secuente y buscar un modelo para la negaci'on.

Sea $\{\alpha_1 \dots \alpha_n\} \vdash \{\beta_1 \dots \beta_m\}$ el secuente bajo an'alisis.
La negaci'on se puede escribir como:

\begin{equation}
\bigwedge\limits_{i=1}^n{\alpha_i} \wedge \bigwedge\limits_{j=1}^m{\neg \beta_{j}}
\end{equation}

Con lo cual para armar la lista de f'ormulas de entrada, para cada $\alpha_{i}$ del antecedente se agrega $\alpha_{i}$ como axioma y para cada $\beta_{i}$ del consecuente se agrega $\neg \beta_{i}$ tambi'en como axioma.

Encontrar un modelo para una especificaci'on armada de 'este modo implica la existencia de un contraejemplo para el secuente procesado.


\subsubsection{Integraci'on con Heterogenius}

Como en el caso de los demostradores autom'aticos fue necesario \textit{refactorizar} el diseño de los buscadores de contraejemplos para permitir una mayor flexibilidad a la hora de agregar nuevas herramientas con soporte de \textit{TPTP-FOF}, ver fig. \ref{diagramaclasesbuscadorescontraejemplos}.

\begin{figure}[]
	\includegraphics[width=450px, angle=90]{img/arq_ce.png}
	\centering
	\caption{Buscadores de contraejemplos.}
	\label{diagramaclasesbuscadorescontraejemplos}
\end{figure}

El diseño de los buscadores de contraejemplos basados en \textit{TPTP-FOF} es an'alogo a los calculadores de secuentes. Se usa la misma clase \textbf{ExternalToolRunner} que en el caso anterior, para la abstracci'on de la ejecuci'on de la herramienta en s'i. Pero en lugar de subclasificar \textbf{ProverParser} se subclasifica la clase abstracta \textbf{ModelFinderParser} que tiene un comportamiento an'alogo, o sea analiza e interpreta la salida \textit{STDOUT} de la herramienta correspondiente.

Al igual que en el caso anterior, existe una herramienta que coordina el proceso de b'usqueda de contraejemplos que es \textbf{FOFModelFinder}.