\section{Extensi'on de Heterogenius con Herramientas de L'ogica de Primer Orden}

Existen numerosas herramientas que funcionan con el lenguaje de l'ogica de primer orden \textit{TPTP-FOF}. Para permitir la integraci'on de estas herramientas con Heterogenius y abrir el camino para la interacci'on con las futuras tecnologias basadas en 'este lenguaje, se agreg'o \textit{TPTP-FOF} al motor de Heterogenius como un lenguaje de an'alisis.

Junto a la integraci'on de \textit{TPTP-FOF}, se incorporaron los siguientes mecanismos para poder usar las herramientas correspondientes:

\begin{itemize}
\item Se permiti'o la carga de especificaciones escritas puramente en \textit{TPTP-FOF}.

\item Se agreg'o una $\rho-$translation desde las formulas \textit{PDOCFA} a \textit{TPTP-FOF}. [TODO: referenciar la seccion que explica esto en detalle].

\end{itemize}

Teniendo el soporte de \textit{TPTP-FOF} por parte del motor de c'alculo de secuentes de Heterogenius, integramos algunas de las herramientas mas difundidas en el 'ambito de demostradores autom'aticos de teoremas para el lenguaje \textit{TPTP-FOF}: \textit{E-Prover} y \textit{SPASS} como calculadores de secuentes; \textit{E-Prover} y \textit{Mace4} como buscadores de contraejemplos.

\subsection{Herramientas Usadas}

\subsubsection{E-Prover}

Es un demostrador autom'atico de teoremas de l'ogica de primer orden basado en el calculo por superposici'on. Adem'as realiza b'usquedas de modelos por lo cual tambi'en se usa como un buscador de contraejemplos.

\subsubsection{SPASS}

Es un demostrador autom'atico de teoremas de l'ogica de primer orden con igualdad desarrollado por el Instituto Max Planck.

Desde el 2000, tanto \textit{SPASS} como \textit{E-Prover} ocupan los primeros lugares en la competencia anual de demostradores de teoremas \textit{CASC} (CADE ATP System Competition).

\subsubsection{Mace4}

Es un buscador de modelos finitos y contraejemplos para l'ogica de primer orden. Se seleccion'o debido a su notoriedad en el 'ambito de herramientas autom'aticas de l'ogica de primer orden.

\subsection{C'alculo de secuentes}

Al introducir el soporte para lenguajes de primer 'orden y herramientas autom'aticas fue necesario extender el c'alculo de secuentes agregando algunas reglas nuevas.

Sea $\Gamma \vdash \alpha$ el secuente que se quiere analizar, se introducen las siguientes reglas:

\begin{prooftree}
\LeftLabel{\textbf{Regla 1:}}
\AxiomC{$\Gamma \vdash \alpha$}
\RightLabel{(si vale $\Gamma \vdash^{fof} \alpha$)}
\UnaryInfC{$\top$}
\end{prooftree}

'Esta regla indica que si se logra encontrar una demostraci'on del secuente $\Gamma \vdash \alpha$ con un demostrador autom'atico \textit{TPTP-FOF}, el resultado es $\top$ y se termina la demostraci'on.


\begin{prooftree}
\LeftLabel{\textbf{Regla 2:}}
\AxiomC{$\Gamma \vdash \alpha$}
\RightLabel{(si existe $\mathcal{M} \in Mod^{fof}(\Gamma$) y $\mathcal{M} \nvDash^{fof} \alpha$ )}
\UnaryInfC{$\bot$}
\end{prooftree}

Si de lo contrario, se logra encontrar un modelo de $\Gamma$ que no satisface $\alpha$, 'este modelo es un contraejemplo y el resultado del an'alisis es $\bot$.


\begin{prooftree}
\LeftLabel{\textbf{Regla 3:}}
\AxiomC{$\Gamma \vdash \alpha$}
\RightLabel{(si no)}
\UnaryInfC{$\Gamma \vdash \alpha$}
\end{prooftree}

Por 'ultimo, debido a que la l'ogica de primer orden no es completa y la ejecuci'on de las herramientas autom'aticas est'a limitada por un timeout, un caso posible es el de no encontrar ni una demostraci'on ni un contraejemplo. En 'este caso no se sabe nada y el resultado es el mismo secuente.

Al usar alg'un demostrador autom'atico de teoremas se aplican las reglas 1 y 3. En cambio cuando se realiza una b'usqueda de contraejemplo, las reglas usadas son 2 y 3.

%á