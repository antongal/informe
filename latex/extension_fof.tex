\section{Extendiendo Heterogenius con demostradores de primer orden}

Desde el a\~no 1996 se celebra anualmente la competencia de demostradores autom'aticos \emph{CADE ATP System Comptetion} \cite{casc} como parte de la conferencia internacional sobre deducci'on autom'atica CADE.
Dicha competencia es uno de los principales eventos donde varios de los m'as famosos grupos de investigaci'on que trabajan en el desarrollo de herramientas de demostraci'on autom'atica se reunen para evaluar la \emph{performance} de sus sistemas.
Gracias a ello, un gran n'umero de herramientas de demostraci'on autom'atica aceptan entradas escritas en el lenguaje propio de la competencia: \textit{TPTP-FOF} \cite{fof}.

Es por ello que hemos decidido agregar el mencionado lenguaje al motor de Heterogenius, de modo de permitir la integraci'on con herramientas actuales y abrir el camino para la interacci'on con las futuras tecnologías que soporten TPTP-FOF.

Junto a la integraci'on de TPTP-FOF, se incorporaron los siguientes mecanismos para poder usar las herramientas correspondientes.

\begin{itemize}
\item Se permiti'o la carga de especificaciones escritas puramente en TPTP-FOF.

\item Se agreg'o una traducción $\rho$ desde el lenguaje de \textit{PDOCFA} a \textit{TPTP-FOF}. 

\item Se agregaron acciones espec'ificas que permiten invocar a las herramientas en cuesti'on.
\end{itemize}

\subsection{Traducci'on de \emph{PDOCFA} a \emph{TPTP-FOF}}

...\todomm{Esto lo escribo yo}

\subsection{Herramientas incorporadas}

Teniendo el soporte de \textit{TPTP-FOF} por parte del motor de c'alculo de secuentes de Heterogenius, integramos algunas de las herramientas mas difundidas en el 'ambito de demostradores autom'aticos de teoremas para el lenguaje \textit{TPTP-FOF}: \textit{SPASS}\todomm{AGregar referencia} como calculador de secuentes\todomm{Explicar qu'e es un calculador de secuentes en el cap'itulo de preliminares.}, \textit{Mace4}\todomm{Agregar referencia} como buscadores de contraejemplos, y \textit{E-Prover} \todomm{Agregar referencia} 

\subsubsection{E-Prover}

Es un demostrador autom'atico de teoremas de l'ogica de primer orden basado en el calculo por superposici'on\todomm{habr'ia que dar una explicaci'on muy b'asica de c'omo es este an;alisis, tipo wikipedia, pero al menos mencionarlo en la secci'on de preliminares}. Adem'as realiza b'usquedas de modelos por lo cual tambi'en se usa como un buscador de contraejemplos.

\subsubsection{SPASS}

Es un demostrador autom'atico de teoremas de l'ogica de primer orden con igualdad desarrollado por el Instituto Max Planck.\todomm{Explicar (al menos mencionar) en qu'e m'etodo de an'alisis est'a basado  (como se menciona para e-prover).}

Desde el 2000, tanto \textit{SPASS} como \textit{E-Prover} ocupan los primeros lugares en la competencia anual de demostradores de teoremas \textit{CASC} (CADE ATP System Competition).

\subsubsection{Mace4}

Es un buscador de modelos finitos y contraejemplos para l'ogica de primer orden. Se seleccion'o debido a su notoriedad en el 'ambito de herramientas autom'aticas de l'ogica de primer orden.\todomm{Notoriedad en qu'e sentido? Explicar (al menos mencionar) en qu'e m'etodo de an'alisis est'a basado.}

\subsection{C'alculo de secuentes} \todomm{Esta subsecci'on hay que reescribirla. Esto no se deber'ia presentar como una modificaci'on al c'alculo de secuentes, sino como la introducci'on de nuevas acciones de demostraci'on de heterogenius.}

Al introducir soporte para lenguajes de primer 'orden y herramientas autom'aticas fue necesario extender el c'alculo de secuentes agregando algunas reglas nuevas.

Sea $\Gamma \vdash \alpha$ el secuente que se quiere analizar, se introducen las siguientes reglas:

\begin{prooftree}
\LeftLabel{\textbf{Regla 1:}}
\AxiomC{$\Gamma \vdash \alpha$}
\RightLabel{(si vale $\Gamma \vdash^{fof} \alpha$)}
\UnaryInfC{$\top$}
\end{prooftree}

'Esta regla indica que si se logra encontrar una demostraci'on del secuente $\Gamma \vdash \alpha$ con un demostrador autom'atico \textit{TPTP-FOF}, el resultado es $\top$ y se termina la demostraci'on.


\begin{prooftree}
\LeftLabel{\textbf{Regla 2:}}
\AxiomC{$\Gamma \vdash \alpha$}
\RightLabel{(si existe $\mathcal{M} \in Mod^{fof}(\Gamma$) y $\mathcal{M} \nvDash^{fof} \alpha$ )}
\UnaryInfC{$\bot$}
\end{prooftree}

Si de lo contrario, se logra encontrar un modelo de $\Gamma$ que no satisface $\alpha$, 'este modelo es un contraejemplo y el resultado del an'alisis es $\bot$.


\begin{prooftree}
\LeftLabel{\textbf{Regla 3:}}
\AxiomC{$\Gamma \vdash \alpha$}
\RightLabel{(si no)}
\UnaryInfC{$\Gamma \vdash \alpha$}
\end{prooftree}

Por 'ultimo, debido a que la l'ogica de primer orden no es completa y la ejecuci'on de las herramientas autom'aticas est'a limitada por un timeout, un caso posible es el de no encontrar ni una demostraci'on ni un contraejemplo. En 'este caso no se sabe nada y el resultado es el mismo secuente.

Al usar alg'un demostrador autom'atico de teoremas se aplican las reglas 1 y 3. En cambio cuando se realiza una b'usqueda de contraejemplo, las reglas usadas son 2 y 3.

