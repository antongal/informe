\section{Extendiendo Heterogenius con demostradores de primer orden}

Tal como se mencionó en el capítulo \ref{capitulo:TPTP-FOF}, la competencia \textit{CASC} ha adquirido una gran reputación en los últimos años.

Junto a la integraci'on de TPTP-FOF, se incorporaron los siguientes mecanismos para poder usar las herramientas correspondientes.

\begin{itemize}
\item Se permiti'o la carga de especificaciones escritas puramente en TPTP-FOF.

\item Se agreg'o una traducción $\rho$ desde el lenguaje de \textit{PDOCFA} a \textit{TPTP-FOF}. 

\item Se agregaron acciones espec'ificas que permiten invocar a las herramientas en cuesti'on.
\end{itemize}

\subsection{Traducci'on de \emph{PDOCFA} a \emph{TPTP-FOF}}

...\todomm{Esto lo escribo yo}

\subsection{Herramientas incorporadas}

Teniendo el soporte de \textit{TPTP-FOF} por parte del motor de c'alculo de secuentes de Heterogenius, integramos algunas de las herramientas mas difundidas en el 'ambito de demostradores autom'aticos de teoremas para el lenguaje \textit{TPTP-FOF}: \textit{SPASS} \cite{spass} y \textit{E-Prover} \cite{s13} como calculadores de secuentes y \textit{Mace4} \cite{m05} y \textit{E-Prover} como buscadores de contraejemplos.

\subsubsection{SPASS}

Desde el 2000, tanto \textit{SPASS} como \textit{E-Prover} ocupan los primeros lugares en la competencia anual de demostradores de teoremas \textit{CASC} (CADE ATP System Competition).

\subsection{Modificaciones al c'alculo de Heterogenius}

Al introducir soporte para lenguajes de l'ogica de primer 'orden y herramientas autom'aticas de an'alisis de estos lenguajes, fue necesario tambi'en extender las acciones de demostraci'on de Heterogenius agregando tres reglas nuevas.

Sea $\Gamma \vdash \alpha$ el secuente que se quiere analizar, se introducen las siguientes reglas:

\begin{prooftree}
\LeftLabel{\textbf{Regla 1:}}
\AxiomC{$\top$}
\RightLabel{\qquad si vale $\Gamma \vdash^{fof} \alpha$}
\UnaryInfC{$\Gamma \vdash \alpha$}
\end{prooftree}

Esta regla indica que si se logra encontrar una demostraci'on del secuente $\Gamma \vdash \alpha$ con un demostrador autom'atico de primer orden, la rama se considera cerrada.


\begin{prooftree}
\LeftLabel{\textbf{Regla 2:}}
\AxiomC{$\bot$}
\RightLabel{\qquad si existe $\mathcal{M} \in Mod^{fof}(\Gamma$) y $\mathcal{M} \nvDash^{fof} \alpha$ }
\UnaryInfC{$\Gamma \vdash \alpha$}
\end{prooftree}

Si de lo contrario, se logra encontrar un modelo de $\Gamma$ que no satisface $\alpha$, este modelo es un contraejemplo y la rama contin'ua abierta.


\begin{prooftree}
\LeftLabel{\textbf{Regla 3:}}
\AxiomC{$\Gamma \vdash \alpha$}
\RightLabel{(si no)}
\UnaryInfC{$\Gamma \vdash \alpha$}
\end{prooftree}

Por 'ultimo, debido a que la l'ogica de primer orden no es completa y la ejecuci'on de las herramientas autom'aticas est'a limitada por un \textit{timeout}, es posible no encontrar ni una demostraci'on ni un contraejemplo. En 'este caso no se puede concluir nada y el resultado es el mismo secuente.

Con 'estas reglas se cubren todos los casos, tanto de demostraci'on como los de refutaci'on de una f'ormula de primer 'orden. Al usar alg'un demostrador autom'atico de teoremas se aplican las reglas 1 y 3. En cambio cuando se realiza una b'usqueda de contraejemplo, las reglas usadas son 2 y 3.

