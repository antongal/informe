\chapter{Introducci'on}
\section{Motivaci'on}

Analisis de software. \\
Heterogeneidad. \\
Tool support: lightweight y heavyweight. \\
Usabilidad e interfaces. \\

\section{Objetivos Especificos}

\subsection{Extensi'on de Heterogenius}

El lenguaje de l'ogica de primer orden TPTP-FOF, al ser muy difundido en la comunidad de investigadores de demostradores autom'aticos de teoremas es soportado por numerosas herramientas. Entre ellas EProver y SPASS, demostradores autom'aticos de teoremas; EProver y Mace4, buscadores de modelos. 

Para permitir el uso de todas 'estas herramientas y otras, se decidi'o integrar TPTP-FOF con Heterogenius mediante la implementaci'on de una $\rho-$translation desde el lenguaje $PDOCFA$.

\subsection{True Heterogeneity}

Heterogenius en su primera versi'on permiti'o realizar demostraciones heterogeneas mediante traducciones de secuentes. Cada secuente ten'ia f'ormulas de un mismo lenguaje haciendo que los secuentes sean en realidad homogeneos.

Se decidi'o ampliar el concepto de heterogeneidad expandiendolo tambi'en a los secuentes. De esta forma s peermiti'o tener mayor flexibilidad en las demostraciones al poder soportar f'ormulas de distintos lenguajes en el mismo secuente.

\subsection{Ampliaci'on del 'arbol de an'alisis}

Con el objetivo de permitir documentar todo el proceso de demostraci'on (caminos alternativos tomados, decisiones que no produjeron ning'un resultado exitoso, etc), se decidi'o ampliar el concepto de 'arbol de analisis. Para 'esto se incluy'o un nuevo tipo de ramificaci'on, que permite soportar ramas de demostraciones alternativas y ramas de decisiones no exitosas.

[TODO: tal vez algun grafico mostrando las diferentes ramas.]