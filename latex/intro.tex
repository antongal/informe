\chapter{Introducci'on}
\section{Motivaci'on}

Si bien el testing es la técnica más frecuentemente utilizada como validación de corrección del software, no es posible alcanzar resultados contundentes mediante su aplicación. En la amplia mayoría de los casos, esta validación parcial es suficiente. Sin embargo, en el desarrollo de sistemas críticos es necesario contar con técnicas que permitan alcanzar mayores grados de certeza sobre la satisfacción de cierto conjunto de propiedades.

En tales contextos se suelen utilizar herramientas basadas en fuertes fundamentos matemáticos y que pertenecen a una categoría que usualmente se conoce como métodos formales tales como los demostradores de teoremas, con los que se puede lograr certeza absoluta sobre las satisfacción de las propiedades deseadas. No obstante, una contra importante de estos métodos es su falta de automaticidad. Es conocido que la tarea de demostrar propiedades requiere de capacidades matemáticas no tan comunes en equipos de desarrollo de sistemas de software, lo que dificulta la realización de tareas de verificación por estos medios. Esta desventaja está también ligada al poder expresivo del lenguaje utilizado: a mayor expresividad del lenguaje, las herramientas de análisis brindan menos posibilidades de automaticidad.

Un ejemplo clásico de esta correlación está dado por los lenguajes de primer orden. Su expresividad impide que se pueda construir un programa que sea capaz de decidir sobre la verdad o falsedad de una sentencia cualquiera. Sin embargo, se han desarrollado diversas herramientas que son capaces de realizar análisis automáticos para una gran cantidad de casos (como Mace4 \cite{m05}, E \cite{s13}, SPASS \cite{WDFKSW09}, etc.). 

Por otro lado, la complejidad actual de los sistemas informáticos fuerza a los equipos de desarrollo a distribuir la tarea de relevar los requerimientos referidos a distintos aspecto de estos sistemas. Es sabido que diferentes aspectos de un sistema de software son mejor capturados por lenguajes de diferentes características. Esta apreciación vale tanto para lenguajes semiformales como la enorme diversidad de lenguajes diagramáticos de UML \cite{BRJ98}, como para lenguajes formales en donde diferentes lógicas resultan apropiadas para poner de relieve diferentes comportamientos de un sistema informático (algunos ejemplos son, linear temporal logics, tanto su versión proposicional, de primer orden o con operadores de pasado \cite{Pnu77}, \cite{MP95}, para caracterizar propiedades de las ejecuciones de un sistema, lógicas dinámicas, también en sus versiones proposicional y de primer orden \cite{HKT00}, etc.). Si bien esta práctica de construir especificaciones heterogéneas facilita la comprensión y documentación de los sistemas, impone una nueva dificultad a los métodos de análisis, que usualmente se aplican sobre un lenguaje único.

Para salvar esta dificultad, desde hace ya algunos años, se ha estado trabajando en el desarrollo de fundamentos formales y herramientas de análisis de sistemas descriptos en forma heterogénea. Del lado de los fundamentos formales encontramos los trabajos seminales en el campo de la especificación algebraica de software como son \cite{GB84}, \cite{GB92}, \cite{Tar96}, \cite{Dia02}, mientras que del lado de las herramientas se observan trabajos como \cite{Mos02}, \cite{DF96}, \cite{DF02}, \cite{LF06}. 

Otra dimensión de la heterogeneidad es la enorme diversidad de herramientas de análisis que existen hoy en día para cada lenguaje. Así, dado un lenguaje de especificación o diseño de software particular, encontramos un verdadero ecosistema de herramientas que nos brindan la posibilidad de aplicar diferentes técnicas de validación y verificación de propiedades descriptas en ese lenguaje y por lo tanto, es necesario brindar fundamentos formales a un concepto de análisis que permita la interrelación, estructurada formalmente, de estas técnicas. Con Heterogenius \cite{heterogenius} exploramos la construcción de esta herramienta cuya racionalización formal tiene origen en $\argentum$ \cite{frias:relmics01}.

Heterogenious es hoy una herramienta que integra gran variedad de lenguajes, vinculados estos por medio de traducciones que preservan la semántica de las especificaciones, y facilitando el uso de diversas herramientas que permiten el análisis de propiedades en dichos lenguajes. Actualmente, una enorme variedad de herramientas, tanto demostradores semiautomáticos como bounded model-checkers automáticos, han elegido TPTP \cite{tptp} como lenguaje nativo para la realización del análisis lo que habilita el uso de una enorme variedad de métodos automáticos y semiautomáticos de análisis para especificaciones que pueden ser escritas en este lenguaje u otro para el cual existe un traducción que preserva la semántica de las fórmulas a este.\\

Las contribuciones principales de esta tesis, que pueden leerse a modo de objetivos específicos del trabajo, son los siguientes:

\begin{itemize}
\item \textbf{Extensi'on del 'arbol de an'alisis:} ya en Heterogenius habíamos presentado como concepto la noción de árbol de análisis. Esto es una generalización del concepto de árbol de prueba para un lenguaje lógico, pero que soporta la invocación a herramientas auxiliares, como por ejemplo, buscadores de modelos o contraejemplos, que si bien no configuran una transformación del secuente en cuestión, permite documentar el proceso de análisis. Si bien esta ampliación del concepto de árbol de prueba es una ganancia metodológica respecto de la utilización disociada de las herramientas disponibles para un cierto lenguaje, todavía es posible realizar ciertas mejoras vinculadas a una interpretación aun más general del concepto de análisis. Uno de estos elementos es la posibilidad de documentar en mayor profundidad el proceso de análisis a partir de tomar en cuenta los caminos alternativos tomados, decisiones que no produjeron ning'un resultado exitoso, etc.

Para abordar esto se incluy'o un nuevo tipo de ramificaci'on, que permite soportar demostraciones alternativas, decisiones no exitosas, etc.

\item \textbf{Extensi'on de Heterogenius:} Heterogenius integra, como lenguaje, una variedad de lenguajes y herramientas. Debido al creciente n'umero de herramientas capaces de procesar lenguajes basados en la l'ogica de primer 'orden, y su uso en el an'alisis de software, se decidi'o extender Heterogenious con el objetivo de soportar el lenguaje de primer orden \textit{TPTP-FOF}. 'Este lenguaje est'a ampliamente difundido en la comunidad de investigadores dedicados a la implementación y uso de demostradores autom'aticos, y semi-automáticos) de teoremas y es soportado por numerosas herramientas. Esta extensión abre un enorme abanico de posibilidades de interacci'on entre Heterogenius y software de an'alisis externo. 

Para demostrar la utilidad de 'esta extensi'on se integraron los siguientes demostradores autom'aticos y buscadores de contraejemplos: EProver y SPASS, como demostradores autom'aticos teoremas, y EProver y Mace4, como buscadores de modelos. 

En virtud de que Heterogenius es una plataforma de an'alisis heterogeneo y uno de sus objetivos es permitir la interacci'on f'acil entre distintos lenguajes del sistema, se implement'o una traducción de las fórmulas de las \emph{Point-dense Omega Closure fork Algebras} \cite{LF06} (una extensión de las \emph{Fork Algebras} \cite{frias02}) a fórmulas de \textit{TPTP-FOF}. Si bien esta traducción no preserva totalmente la semántica del lenguaje de esta clase de álgebra, aun así es posible derivar conclusiones de importancia para el análisis de especificaciones de software.


\item \textbf{Heterogeneidad Verdadera:} Heterogenius, en su primera versi'on, fue creado como una herramienta capaz de manipular especificaciones heterogenéneas pero con la limitación de que un secuente que formaba parte de un proceso de análisis debía necesariamente ser homogeneo. De 'esta forma, un secuente solamente pod'ia contener f'ormulas pertenecientes a un único lenguaje, y las traducciones se realizaban sobre todas las f'ormulas del secuente. 'Esto presenta una limitación muy fuerte ya que no permite ejercer un control m'as fino sobre los elementos descriptivos contenidos en los secuentes en virtud de no lograr el suficiente grado de heterogeneidad al lleva a cabo el an'alisis. De alguna forma, solo resultaba admisible la utilización de lenguajes diferentes en tanto estos fueran traducibles a un otro que resulte común.

Debido a esto, se ampli'o el concepto de heterogeneidad de forma de permitir la utilización de f'ormulas de distintos lenguajes en el mismo secuente y cuyas traducciones a otros lenguajes puedan realizarse en forma independiente de otras formulas presentes en el secuente.
\end{itemize}

El presente trabajo está organizado de la siguiente forma: en la Sec.~\ref{xxx} presentaremos...