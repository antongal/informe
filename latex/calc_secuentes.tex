\section{Cálculo de secuentes} \label{calculo-secuentes}

El cálculo de secuentes aparece en la década del `30 siendo los sistemas LK y LJ los primeros c'alculos desarrollados por Gerhard Gentzen \cite{gentzen1935}.

Las demostraciones en el c'alculo de secuentes se construyen en forma de un 'arbol donde sus nodos son secuentes y las aristas son aplicaciones de reglas de inferencia. Un secuente se puede ver como dos listas de f'ormulas y se nota $ \Gamma \vdash  \Delta$. La primer lista se denomina \textit{antecedente} y contiene a las hip'otesis. La segunda lista es el \textit{consecuente} y representa lo que se quiere demostrar.

El secuente puede ser interpretado de esta manera: un secuente $\Gamma \vdash  \Delta$ es válido en un modelo semántico \sem{M}, si 

\begin{equation}
\textrm{\sem{M}} \vDash (\bigwedge_{\gamma \in \Gamma} \gamma \Rightarrow \bigvee_{\delta \in \Delta} \delta )
\notag
\end{equation}

Adem'as un secuente puede ser:

\begin{itemize}

\item{Universalmente v'alido}: si es válido para todo modelo semántico.
\item{Inv'alido}, si existe un modelo semántico \sem{N} que es un \emph{contraejemplo} para dicho secuente y que satisface:

\begin{equation}
\textrm{\sem{N}} \vDash (\bigwedge_{\gamma \in \Gamma} \gamma \wedge \bigwedge_{\delta \in \Delta} \neg\delta )
\notag
\end{equation}

\end{itemize}



Una demostraci'on en el c'alculo de secuentes se realiza aplicando reglas de inferencia sobre los nodos hojas hasta llegar a un secuente trivial de la forma $\alpha \vdash \alpha$ con $\alpha$ una fórmula.

En general las reglas de inferencia tienen la siguiente forma:

\begin{prooftree}
	\AxiomC{$\Gamma_{1} \vdash \Delta_{1} $}
	\AxiomC{$\dots$}
	\AxiomC{$\Gamma_{n} \vdash \Delta_{n} $}
	\RightLabel{\scriptsize ($R$)}
	\TrinaryInfC{$\Gamma \vdash \Delta $}
\end{prooftree}
y se interpretan como, dado un secuente hoja del árbol de demostración de la forma $\Gamma \vdash \Delta$, al aplicar la regla $R$ se modifica el árbol agregando $n$ nuevos nodos de la forma $\Gamma_{i}\vdash\Delta_{i}$ como hijos del secuente $\Gamma \vdash \Delta$.

Algunas de las reglas del c'alculo se presentan a continuaci'on:

\vspace{2em}

\begin{tabularx}{\textwidth}{Xc Xc}
	
	%Right OR
	\AxiomC{$\Gamma \vdash \alpha, \beta, \Delta $}
	\RightLabel{\scriptsize (right $\vee$)}
	\UnaryInfC{$\Gamma \vdash (\alpha \vee \beta), \Delta $}
	\DisplayProof
	
	&
	
	%Right THEN
	\AxiomC{$\Gamma , \alpha \vdash \beta, \Delta $}
	\RightLabel{\scriptsize (right $\rightarrow$)}
	\UnaryInfC{$\Gamma \vdash (\alpha \rightarrow \beta), \Delta $}
	\DisplayProof
	
	\\ & \\
	
	%Right NOT
	\AxiomC{$\Gamma , \alpha \vdash \Delta $}
	\RightLabel{\scriptsize (right $\neg$)}
	\UnaryInfC{$\Gamma \vdash (\neg\alpha) , \Delta $}
	\DisplayProof
	
	&
	
	%Right AND
	\AxiomC{$\Gamma \vdash \alpha, \Delta $}
	\AxiomC{$\Gamma \vdash \beta, \Delta $}
	\RightLabel{\scriptsize (right $\wedge$)}
	\BinaryInfC{$\Gamma \vdash (\alpha \wedge \beta) , \Delta $}
	\DisplayProof
	
\end{tabularx}

\vspace{2em}

Existen diferentes versiones del c'alculo de secuentes, y la elecci'on de cu'al usar depende del lenguaje l'ogico que se est'e utilizando para el an'alisis.