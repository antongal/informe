\section{Demostraciones heterogéneas}
\label{heterogenious proofs}

Hoy en dia existe una gran variedad de lenguajes l'ogicos cada uno con sus caracter'isticas. Dependiendo de su poder expresivo, cada lenguaje puede describir mejor algunos aspectos del sistema pero no otros. Generalmente al elegir especificar un sistema utilizando un solo lenguaje nos topamos con este \textit{tradeoff}.

Naturalmente surge la idea de combinar varios lenguajes y usar cada uno para describir aspectos del sistema de modo de aprovechar mejor las caracter'isticas propias de cada uno. Siguiendo 'este concepto se vuelve necesario poder traducir lo escrito de un lenguaje l'ogico a otro \cite{goguen:jacm-39_1,meseguer:lc87,tarlecki:sadt-rtdts95}.

As'i se puede probar que dados los lenguajes l'ogicos $L'$ y $L$, tales que 

\begin{itemize}
\item $L'$ un lenguaje con c'alculo completo.
\item $L$ un lenguaje que se puede traducir a $L'$.
\item $T_{L \to L'}(\alpha)$ una traducci'on de $\alpha$ del lenguaje $L$ a $L'$ 
\item $T_{L \to L'}(\Gamma)$ una traducci'on de f'ormulas de $\Gamma$ del lenguaje $L$ a $L'$.
\end{itemize}
si es posible demostrar en el lenguaje $L'$ la fórmula $T_{L \to L'} (\alpha)$, entonces $\alpha$ es universalmente válida en la clase de modelos de $L$.

Con lo cu'al se obtiene que
$$\Gamma \models_L \alpha \mbox{ si y sólo si } T_{L \to L'}(\Gamma) \vdash_{L'} T_{L \to L'}(\alpha)$$
Adem'as si $L$ tambi'en posee un c'alculo no necesariamente completo, se obtiene que:
$$\mbox{Si } T_{L \to L'}(\Gamma) \vdash_{L'} T_{L \to L'}(\alpha) \mbox{ entonces } \Gamma \vdash_L \alpha$$

Como consecuencia tiene sentido extender el c'alculo dando soporte a las reglas tanto del c'alculo del lenguaje $L$ como las del c'alculo del lenguaje $L'$ agregando una regla adicional para soportar las traducciones de un lenguaje a otro:

\begin{center}
	\AxiomC{$T_{L \to L'}(\Gamma) \vdash_{L'} T_{L \to L'}(\alpha)$}
	\RightLabel{\scriptsize (translation)}
	\UnaryInfC{$\Gamma \vdash_L \alpha$}
	\DisplayProof
\end{center}

De esta forma es posible realizar demostraciones de problemas definidos en el lenguaje $L$ y luego pasar al lenguaje $L'$ aplicando la regla \textit{translation}. 'Este tipo de c'alculos fue estudiado por Borzyszkowski en \cite{borzyszkowski:tcs-286_2}.