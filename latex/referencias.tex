
\begin{thebibliography}{tesis}

\bibitem{tptp}
	Sutcliffe, G. The TPTP Problem Library and Associated Infrastructure: The FOF and CNF Parts, v3.5.0. Journal of Automated Reasoning. Vol. 43. Num. 4. Pp. 337-362. 2009.

\bibitem{casc}
	The CADE ATP System Competition
	\url{http://www.cs.miami.edu/~tptp/CASC/}

\bibitem{fof}
	FOF: First Order Formula
	\url{http://www.cs.miami.edu/~tptp/TPTP/SyntaxBNF.html}

\bibitem{heterogenius}
	Manuel Gimenez, Mariano Miguel Moscato, Carlos G. Lopez Pombo, and Marcelo F. Frias. Heterogenius: a framework for hybrid analysis of heterogeneous software specifications. In Aguirre and Ribeiro \cite{AR13}, pages 1045–1058. Workshop affiliated to \cite{DM13}. 2013.
	
\bibitem{m05}
	McCune, W., "Prover9 and Mace4", http://www.cs.unm.edu/~mccune/Prover9, 2005-2010. 

\bibitem{s13}
  Schulz, S. System Description: E 1.8, Proceedings of the 19th LPAR, Stellenbosch, 2013, pp. 477-483, LNCS 8312 © Springer Verlag.

\bibitem{spass} Weidenbach C., Dimova D., Fietzke A., Kumar R., Suda M. and Wischnewski P., 2009, SPASS Version 3.5. in 22nd International Conference on Automated Deduction, CADE 2009, LNCS 5663, pp. 140-145.

\bibitem{otter} J. Kalman. Automated Reasoning with Otter. Rinton Press, Princeton, New Jersey,
2001.

\bibitem{BRJ98} Grady Booch, Jim Rumbaugh, and Ivar Jacobson. The unified modeling language user guide. Addison–Wesley Longman Publishing Co., Inc., Boston, MA, USA, 1998.

\bibitem{Pnu77} Amir Pnueli. The temporal logic of programs. In Proceedings of 18th. Annual IEEE Symposium on Foundations of Computer Science \cite{IEE77}, pages 46–57. 

\bibitem{IEE77} IEEE Computer Society. 18th. Annual IEEE Symposium on Foundations of Computer Science, Los Alamitos, CA, USA, 1977. IEEE Computer Society. 

\bibitem{MP95} Zohar Manna and Amir Pnueli. Temporal Verification of Reactive Systems. Springer-Verlag, New York, 1995. 

\bibitem{HKT00} David Harel, Dexter Kozen, and J. Tiuryn. Dynamic logic. Foundations of Computing. MIT Press, Cambridge, MA, USA, 2000.  

\bibitem{GB84} Joseph A. Goguen and Rod M. Burstall. Introducing institutions. In Clarke and Kozen \cite{CK84}, pages 221–256.  

\bibitem{CK84} Edmund M. Clarke and Dexter Kozen, editors. Carnegie Mellon Workshop on Logic of Programs, volume 184 of Lecture Notes in Computer Science. Springer-Verlag, 1984. 

\bibitem{GB92} Joseph A. Goguen and Rod M. Burstall. Institutions: abstract model theory for specification and programming. Journal of the ACM, 39(1):95–146, 1992. 

\bibitem{Tar96} Andrzej Tarlecki. Moving between logical systems. In Haveraaen et al. \cite{HOD96}, pages 478–502.

\bibitem{HOD96} Magne Haveraaen, Olaf Owe, and Ole-Johan Dahl, editors. Selected papers from the 11th Workshop on Specification of Abstract Data Types, joint with the 8th COMPASS Workshop on Recent

\bibitem{Dia02} Razvan Diaconescu. Grothendieck institutions. Applied Categorical Structures, 10(4):383–402, 2002.

\bibitem{NE02} Mogens Nielsen and Uffe Engberg, editors. International Con- ference on Foundations of Software Science and Computation Structures, Lecture Notes in Computer Science, London, UK, 2002. Springer-Verlag.

\bibitem{Mos02} Till Mossakowski. Heterogeneous development graphs and heterogeneous borrowing. In Nielsen and Engberg \cite{NE02}, pages 326–341. 

\bibitem{DF96} Razvan Diaconescu and Kokichi Futatsugi. Logical semantics for CafeOBJ. Technical Report JAIST-IS-RR-96-0024S, Japan Advanced Institute of Science and Technology, 1996. 

\bibitem{DF02} Razvan Diaconescu and Kokichi Futatsugi. Logical foundations of CafeOBJ. Theoretical Computer Science, 285(2):289–318, 2002. 

\bibitem{JV06}  Michael Johnson and Varmo Vene, editors. 11th. International Conference on Algebraic Methodology and Software Technology, AMAST 2006, volume 4019 of Lecture Notes in Computer Science, Kuressaare, Estonia, July, 5–8 2006. Springer-Verlag. 

\bibitem{LF06} Carlos G. Lopez Pombo and Marcelo F. Frias. Fork algebras as a sufficiently rich universal institution. In Johnson and Vene \cite{JV06}, pages 235–247. 

\bibitem{AR13} Nazareno M. Aguirre and Leila Ribeiro, editors. Latin American Workshop on Formal Methods 2013, August 2013. Workshop affiliated to \cite{DM13}. 

\bibitem{DM13} Pedro R. D’Argenio and Hernán Melgratti, editors. 24th International Conference on Concurrency Theory – CONCUR 2013, volume 8052 of Lecture Notes in Computer Science. Springer- Verlag, 2013.

\bibitem{FBL01}
Marcelo~{F.} Frias, Gabriel~{A.} Baum, and Carlos~{G.} {Lopez Pombo}.
\newblock A comparisson of $\mathbf{A_g}$ with {Alloy}.
\newblock In de~Swart \cite{relmics01}, pages 365--377.

\bibitem{relmics01}
Harrie de~Swart, editor.
\newblock {\em 6th. Conference on Relational Methods in Computer Science
  ({RelMiCS}) - {TARSKI}}, Oisterwijk, The Netherlands, October 2001.
  
\bibitem{goguen:cmwlp84}
	Joseph A. Goguen and Rod M. Burstall. Introducing institutions. In Edmund M. Clarke and Dexter Kozen, editors, Proceedings of the Carnegie Mellon Workshop on Logic of Programs, volume 184 of Lecture Notes in Computer Science, pages 221–256. Springer-Verlag, 1984.

\bibitem{goguen:jacm-39_1}
  Joseph A. Goguen and Rod M. Burstall. Institutions: abstract model theory for specification and programming. Journal of the ACM, 39(1):95–146, 1992.

\bibitem{meseguer:lc87}
	Jos'e Meseguer. General logics. In Heinz-Dieter Ebbinghaus, Jos'e Fernandez-Prida, Manuel Garrido, Daniel Lascar, and Mario Rodr'ıguez Artalejo, editors, Proceedings of the Logic Colloquium ’87, volume 129, pages 275–329, Granada, Spain, 1989. North Holland.

\bibitem{tarlecki:sadt-rtdts95}
	Andrzej Tarlecki. Moving between logical systems. In Magne Have- raaen, Olaf Owe, and Ole-Johan Dahl, editors, Selected papers from the 11th Workshop on Specification of Abstract Data Types Joint with the 8th COMPASS Workshop on Recent Trends in Data Type Specification, volume 1130 of Lecture Notes in Computer Science, pages 478–502. Springer-Verlag, 1996.

\bibitem{borzyszkowski:tcs-286_2}
	Tomasz Borzyszkowski. Logical systems for structured specifications. Theoretical Computer Science, 286:197–245, 2002.

\bibitem{gentzen1935}
	G. Gentzen. Untersuchungen uber das logische schließen i. I. Mathema- tische zeitschrift 39, pages 176–210, 1935.
	
\bibitem{bookFoundations}
	Jean H. Gallier. Logic For Computer Science: Foundations of Automatic Theorem Proving. Wiley, 2003.

\bibitem{frias02}
Marcelo~{F.} Frias.
\newblock {\em Fork algebras in algebra, logic and computer science}, volume~2
  of {\em Advances in logic}.
\newblock World Scientific Publishing Co., Singapore, 2002.

\bibitem{smallcnf2001}
	A. Nonnengart and C. Weidenbach. Computing Small Clause Normal Forms. In A. Robinson and A. Voronkov, editors, Handbook of Automated Reasoning, volume I, chapter 5, pages 335–367. Elsevier Science and MIT Press, 2001.

\bibitem{bg94}
	L. Bachmair and H. Ganzinger. Rewrite-Based Equational Theorem Proving with Selection and Simplification. Journal of Logic and Computation, 3(4):217–247, 1994.

\bibitem{aicom2002}
S. Schulz. E – A Brainiac Theorem Prover. Journal of AI Communications, 15(2/3):111–126, 2002.

\bibitem{jsc2003}
Jürgen Avenhaus, Thomas Hillenbrand, and Bernd Löchner. On Using Ground Joinable Equations in Equational Theorem Proving. Journal of Symbolic Computation, 36(1-2):217–233, 2003.

\bibitem{ijcai2001}
A. Riazanov and A. Voronkov. Splitting without Backtracking. In B. Nebel, editor, Proc. of the 17th International Joint Conference on Artificial Intelligence (IJCAI-2001), Seattle, volume 1, pages 611–617. Morgan Kaufmann, 2001.

\bibitem{dk97}
J. Denzinger, M. Kronenburg, and S. Schulz. DISCOUNT: A Distributed and Learning Equational Prover. Journal of Automated Reasoning, 18(2):189–198, 1997. Special Issue on the CADE 13 ATP System Competition.

\bibitem{zhang96} J. Zhang. Constructing finite algebras with FALCON. J. Automated Reasoning, 17(1):1–22,
1996.

\bibitem{zhang95} J. Zhang and H. Zhang. SEM: A system for enumerating models. In Proc. IJCAI-95,
volume 1, pages 298–303. Morgan Kaufmann, 1995.

\bibitem{knuthbendix} D. Knuth, "The Genesis of Attribute Grammars".

\bibitem{alloy}  Jackson, D., Shlyakhter, I., and Sridharan, M., A Micromodularity Mechanism.
Proc. ACM SIGSOFT Conf. Foundations of Software Engineering/European Software
Engineering Conference (FSE/ESEC ’01), Vienna, September 2001.

\bibitem{pdocfa} Dynamite: Alloy Analyzer+PVS in the Analysis and Verification of Alloy Specifications. Frias, Marcelo and Lopez Pombo, Carlos and Moscato, Mariano. 2006.

\bibitem{church} Alonzo Church. A note on the entscheidungsproblem. Journal of Symbolic Logic, 1:40–41, 1936.

\bibitem{smallscope} A Andoni, D Daniliuc, S Khurshid, and D Marinov. Evaluating the
“small scope hypothesis”. 2003.

\bibitem{BH96} A. Buch and Th. Hillenbrand. Waldmeister: Development of a High Performance Completion-Based Theorem Prover. SEKI-Report SR-96-01.

\bibitem{vampire} Laura Kovacs, Andrei Voronkov. First-Order Theorem Proving and Vampire. CAV 2013.

\end{thebibliography}

