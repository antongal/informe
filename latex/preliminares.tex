\chapter{Preliminares}

En esta secci'on se explicar'an los conceptos b'asicos necesarios para entender este trabajo. En particular se va a hablar de los lenguajes \textit{Alloy} y \textit{TPTP-FOF}, el c'alculo de secuentes y el c'alculo por superposici'on, demostraciones heterogeneas y las herramientas utilizadas en 'esta tesis como lo son \textit{AlloyAnalyzer}, \textit{E-Prover}, \textit{SPASS}, \textit{Mace4} y \textit{Heterogenius}. 

En el resto del texto se utilizar'an letras griegas en min'uscula para referirnos a f'ormulas y las letras griegas may'usculas para representar conjuntos o listas de f'ormulas.

\section{Alloy}

\textit{Alloy}\cite{alloy} es un lenguaje de especificaci'on declarativo que permite definir dominios de datos a trav'es de conjuntos de elementos at'omicos, y expresar propiedades sobre las relaciones definidas a partir de esos dominios. Este lenguaje se basa en la l'ogica de primer orden y la extiende con la clausura reflexo-transitiva del operador de composici'on de relaciones binarias, y posee una sem'antica basada en conjuntos y relaciones.

Los modelos Alloy est'an estructurados en m'odulos. Las signaturas (identificadas con la palabra clave \textbf{sig}) denotan dominios de datos. Una signatura puede ser abstracta, en cuyo caso solo contendr'a elementos de las signaturas que heredan de ella. En Alloy la herencia se identifica con la palabra clave \textbf{extends}. Las signaturas pueden contener atributos, los cuales en \textit{Alloy} son llamados \textbf{fields}.

Una vez definidos los dominios, se pueden especificar restricciones sobre ellos. En \textit{Alloy} los axiomas son llamados \textbf{facts}. Para simplificar la escritura de las f'ormulas, se pueden introducir funciones, identificadas como \textbf{fun}, y predicados, notados \textbf{pred}.

Los t'erminos denotan relaciones, y se construyen utilizando signaturas, atributos de signaturas, y constantes tales como \textbf{univ} (el conjunto de todos los objetos en el modelo), \textbf{iden} (la relaci'on binaria identidad definida sobre el conjunto \textbf{univ}), y \textbf{none} (el conjunto vacio).

Utilizando operadores relacionales se pueden construir t'erminos m'as complejos. Contamos con la operaci'on de diferencia, uni'on, intersecci'on y composici'on ($-$, $+$, $\&$ respectivamente). La traspuesta de una relaci'on binaria $R$ (notada en Alloy con $\backsim{R}$) invierte los elementos de las tuplas de $R$. La clausura transitiva y la reflexo-transitiva de relaciones binarias se identifican con  $\hat{}$ y $*$ respectivamente.

Las f'ormulas at'omicas se construyen utilizando igualdades e inclusiones entre t'erminos. Utilizando conectivos y cuantificadores est'andares de la l'ogica de primer orden se puede construir f'ormulas m'as complejas.

Tambi'en se pueden utilizar abreviaturas para simplificar las f'ormulas. Por ejemplo, la palabra clave \textbf{no} fuerza que el t'ermino anotado con ella denote una relaci'on vac'ia. De manera similar, la aparici'on de \textbf{one} implica que el termino denota una relaci'on \textit{singleton} (con s'olo un par). La palabra clave \textbf{lone} hace que un t'ermino denote un conjunto con cardinalidad a lo sumo uno.

Alloy permite especificar propiedades que uno espera que valgan en el modelo, a traves de sentencias llamadas assertions.


\section{Alloy Analyzer}

Los modelos definidos con el lenguaje \textit{Alloy} pueden ser analizados con \textit{Alloy Analyzer}.
Esta herramienta analiza los modelos Alloy autom'aticamente utilizando \textit{SAT}-\textit{solvers} externos, ya sea buscando contraejemplos de una aserci'on determinada, o generando una instancia del modelo.

Para que el \textit{Alloy Analyzer} pueda traducir los modelos \textit{Alloy} a f'ormulas de la l'ogica proposicional, el usuario debe proveer cotas para el tamaño de los dominios de datos. Estas cotas son llamadas \textbf{scopes}.


Como se mencion'o anteriormente el lenguaje \textit{Alloy} extiende a la l'ogica de primer orden, por lo que determinar la veracidad de una aserci'on no es decidible \cite{meseguer:lc87,church}. 

Resulta que, el an'alisis mediante \textit{SAT}-\textit{solvers} es v'alido 'uniicamente para las estructuras sem'anticas cuyos dominios de datos cumplen con las cotas establecidas mediante los \textit{scopes}. Es decir, si no se encuentra un contraejemplo, puede de todas maneras existir uno si se consideran cotas m'as grandes. Aunque son evidentes las limitaciones de esta t'ecnica de an'alisis, suele resultar muy 'util a la hora de escribir modelos. Seg'un la \textit{small scope hypothesis} \cite{smallscope} una gran proporci'on de los errores introducidos en un modelo pueden encontrarse en instancias pequeñas.

Con lo cual, el tipo de an'alisis ofrecido por el \textit{AlloyAnalyzer} es adecuado en la mayor'ia de los casos. Sin embargo en algunas situaciones puede no ser suficiente, por ejemplo en el modelado de sistemas cr'iticos. Se han diseñado e implementado herramientas que permiten complementar el an'alisis de \textit{SAT-solving} con la demostraci'on interactiva de teoremas \textit{Alloy}. Por ejemplo, \textit{Dynamite} \cite{pdocfa} y \textit{Heterogenius} \cite{heterogenius} son herramientas que permiten utilizar c'alculo de secuentes para construir las demostraciones de las aserciones de \textit{Alloy}.

En el contexto de \textit{Heterogenius}, \textit{AlloyAnalyzer} se utiliza como un buscador de contraejemplos. Esto se logra negando la f'ormula que se quiere demostrar y utilizando a \textit{AlloyAnalyzer} para buscar un modelo. Si se puede encontrar un modelo, entonces dicho modelo resulta el contraejemplo de la especificaci'on analizada. Si no se logra encontrar uno, no se puede asegurar la validez de la especificaci'on.

\section{Cálculo de secuentes} \label{calculo-secuentes}

El cálculo de secuentes es una familia de sistemas de demostración. Los primeros cálculos de secuentes (los sistemas LK y LJ) fueron desarrollados originalmente por Gerhard Gentzen en la década del `30\cite{gentzen1935}.

En el cálculo de secuentes una demostración se construye como un árbol en el que sus nodos son secuentes.
Un secuente es un par ordenado de listas finitas de fórmulas, y lo notamos $ \Gamma \vdash  \Delta$.
Su interpretación intuitiva puede pensarse de esta manera: un secuente $\Gamma \vdash  \Delta$ es válido en un modelo semántico \sem{M}, si $\textrm{\sem{M}} \vDash (\bigwedge_{\gamma \in \Gamma} \gamma \Rightarrow \bigvee_{\delta \in \Delta} \delta )$.
Luego, un secuente es \emph{universalmente válido} (o simplemente \emph{válido}) si es válido para todo modelo semántico.
Correspondientemente, un secuente $\Gamma \vdash  \Delta$ es \emph{inválido} si existe un modelo semántico \sem{N} tal que $\textrm{\sem{N}} \vDash (\bigwedge_{\gamma \in \Gamma} \gamma \wedge \bigwedge_{\delta \in \Delta} \neg\delta )$.
En tal caso, diremos que \sem{N} es un \emph{contraejemplo} para ese secuente.

Para demostrar la fórmula $\phi$, se parte de un árbol con un único nodo: $ \vdash \phi $ (un abuso de notación para el secuente $ \emptyset \vdash \{\phi\} $). Mediante la aplicación de reglas de inferencia sobre las hojas, el árbol crece en altura incorporando nuevos secuentes como hijos del nodo sobre el cual se aplica la regla.
La demostración se considera terminada cuando todas las hojas del árbol son de la forma $\alpha \vdash \alpha$ con $\alpha$ una fórmula, los cuales son llamados \emph{secuentes triviales} o \emph{secuentes terminales}.
%con $ \Sigma \cap \Gamma = \emptyset $ 

Las reglas de inferencia tienen la forma que se muestra a continuación,
\begin{prooftree}
	\AxiomC{$\Gamma_{1} \vdash \Delta_{1} $}
	\AxiomC{$\dots$}
	\AxiomC{$\Gamma_{n} \vdash \Delta_{n} $}
	\RightLabel{\scriptsize ($R$)}
	\TrinaryInfC{$\Gamma \vdash \Delta $}
\end{prooftree}
y se pueden interpretar de esta manera: dada una hoja del árbol de demostración de la forma $\Gamma \vdash \Delta$, la aplicación de la regla $R$ modifica el árbol agregando $n$ nuevas hojas de la forma $\Gamma_{i}\vdash\Delta_{i}$ como hijas del secuente $\Gamma \vdash \Delta$\footnote{Si bien las reglas pueden interpretarse en sentido inverso, preferimos esta dirección debido a que vamos a darle más atención al proceso de construcción de la demostración que a su lectura posterior.}.

\bigskip

A modo de ejemplo, veamos cómo construir la demostración de la fórmula proposicional $(p \vee \neg p) \wedge (q \rightarrow q)$ utilizando un cálculo de secuentes para esa lógica.
La fórmula dada es una conjunción de tautologías, ya que $(p \vee \neg p)$ y $(q \rightarrow q)$ son válidas para toda valuación, y por lo tanto es ella misma una tautología.
Luego, por la compleción del cáculo, existe una demostración como las descriptas más arriba para esta fórmula.


En la figura \ref{reglasProposicional} pueden verse algunas reglas del cálculo que nos serán útiles en la demostración.
%~ \bigskip

\begin{figure}
\begin{tabularx}{\textwidth}{Xc Xc}
	
	%Right OR
	\AxiomC{$\Gamma \vdash \alpha, \beta, \Delta $}
	\RightLabel{\scriptsize (right $\vee$)}
	\UnaryInfC{$\Gamma \vdash (\alpha \vee \beta), \Delta $}
	\DisplayProof
	
	&
	
	%Right THEN
	\AxiomC{$\Gamma , \alpha \vdash \beta, \Delta $}
	\RightLabel{\scriptsize (right $\rightarrow$)}
	\UnaryInfC{$\Gamma \vdash (\alpha \rightarrow \beta), \Delta $}
	\DisplayProof
	
	\\ & \\
	
	%Right NOT
	\AxiomC{$\Gamma , \alpha \vdash \Delta $}
	\RightLabel{\scriptsize (right $\neg$)}
	\UnaryInfC{$\Gamma \vdash (\neg\alpha) , \Delta $}
	\DisplayProof
	
	&
	
	%Right AND
	\AxiomC{$\Gamma \vdash \alpha, \Delta $}
	\AxiomC{$\Gamma \vdash \beta, \Delta $}
	\RightLabel{\scriptsize (right $\wedge$)}
	\BinaryInfC{$\Gamma \vdash (\alpha \wedge \beta) , \Delta $}
	\DisplayProof
	
\end{tabularx}
\caption{Algunas reglas del cálculo de secuentes para lógica proposicional.}\label{reglasProposicional}
\end{figure}
%~ \bigskip

Inicialmente tenemos, como raíz del árbol de demostración, al secuente que no tiene antecedentes y sólo tiene como consecuente la fórmula que queremos demostrar: $$ \vdash (p \vee \neg p) \wedge (q \rightarrow q)$$
Aplicando la regla (\textit{right} $\wedge$) se obtienen dos secuentes hijos, los cuales indican nuestras nuevas obligaciones de demostración:
$$  \vdash (p \vee \neg p) \;\;\;\;\;\;\; \vdash (q \rightarrow q) $$
Ataquemos primero el segundo secuente recién obtenido. Al aplicar la regla (\textit{right} $\rightarrow$), se obtiene un único secuente hijo: $$ q \vdash q $$
Como este es un secuente trivial, ya podemos considerar cerrada esta rama de la demostración.
Volvamos entonces al secuente $\vdash (p \vee \neg p)$. Aplicando la regla (\textit{right} $\vee$), obtenemos un único secuente hijo:
$$ \vdash p \; , \; \neg p $$
Finalmente aplicando la regla (\textit{right} $\neg$) obtenemos nuevamente un único secuente hijo $$ p \vdash p $$
Dado que es un secuente trivial, también podemos considerar cerrada esta rama de la demostración.
Así llegamos a obtener un árbol con raíz $ (p \vee \neg p) \wedge (q \rightarrow q)$ y cuyas hojas son todas secuentes triviales.
Luego, hemos demostrado dicha fórmula.
En la figura \ref{arbolDemoEjemplo} se muestra la demostración anterior escrita utilizando la notación tradicional del cálculo de secuentes.

\begin{figure}
%\missingfigure{Arbol de demostración tradicional}

\begin{prooftree}
	\AxiomC{$p \vdash p$}
	\RightLabel{\scriptsize(right $\neg$)}	
		\UnaryInfC{$\vdash p , \neg p$}
	\RightLabel{\scriptsize(right $\vee$)}	
		\UnaryInfC{$\vdash p \vee \neg p$}

	\AxiomC{$q \vdash q $}	
	\RightLabel{\scriptsize(right $\rightarrow$)}	
	\UnaryInfC{$\vdash (q \rightarrow q)$}

\RightLabel{\scriptsize(right $\wedge$)}
\BinaryInfC{$\vdash (p \vee \neg p) \wedge (q \rightarrow q)$}
\end{prooftree}
\caption{Ejemplo de demostración en notación tradicional}\label{arbolDemoEjemplo}
\end{figure}


%\begin{prooftree}
%\AxiomC{$P$}
%\RightLabel{\scriptsize(1)}
%\UnaryInfC{$\neg P$}
%
%%             \BinaryInfC{$\bot$}
%%             \RightLabel{\scriptsize(1)}
%%             \UnaryInfC{$\neg\neg P$}
%\end{prooftree}

\bigskip

%\mtodo{}{Este párrafo debería ir después del ejemplo; como está ahora, la frase "A modo de ejemplo" del párrafo siguiente parece referirse a la compleción y corrección del cálculo.}
%\todo{Ya lo moví, pero lo sacaría la verdad}
Para las lógicas más tradicionales (lógica proposicional, lógica de primer orden, etc.) puede demostrarse que este cáculo es completo y correcto (ver \cite{bookFoundations}, por ejemplo), es decir que para toda fórmula semánticamente válida puede construirse una demostración y que toda fórmula para la que pueda construirse una demostración es semánticamente válida.

\bigskip

Vale aclarar que, aunque se pueden hallar distintas versiones de estos cálculos aún para una misma lógica, cualquier cálculo de secuentes para lógica proposicional debe poseer más reglas que las utilizadas en el ejemplo anterior.
Por lo menos debe proveer las correspondientes a los mismos símbolos lógicos que aparecen en la figura \ref{reglasProposicional} pero para el lado izquierdo del $\vdash$.
Si bien con esas reglas puede demostrarse la corrección y compleción del cálculo (ver \cite{bookFoundations}, por ejemplo) algunos autores también incluyen otras reglas que aunque supérfluas desde el punto de vista teórico, son muy útiles al utilizar el cálculo para construir demostraciones.
Estas reglas tienen que ver con la composición estructural de los secuentes.
En la figura \ref{reglasEstructurales}, pueden verse algunos ejemplos de estas reglas: (\emph{left} \textbf{C}) y (\emph{right} \textbf{C}) permiten duplicar reglas del antecedente y del consecuente, (\emph{left} \textbf{X}) y (\emph{right} \textbf{X}) permiten cambiar el orden de las fórmulas, y la \textit{cut rule} puede interpretarse como la introducción de una nueva hipótesis a la demostración o como una separación en casos basada en el cumplimiento de la fórmula $\alpha$.

\begin{figure}
\begin{center}
\begin{tabularx}{0.7\textwidth}{Xc Xc}

	\AxiomC{$\Gamma_1, \alpha, \alpha, \Gamma_2 \vdash \Delta $}
	\RightLabel{\scriptsize (left \textbf{C})}
	\UnaryInfC{$\Gamma_1, \alpha, \Gamma_2 \vdash  \Delta $}
	\DisplayProof
	
	&

	\AxiomC{$\Gamma \vdash \Delta_1, \alpha, \alpha, \Delta_2 $}
	\RightLabel{\scriptsize (right \textbf{C})}
	\UnaryInfC{$\Gamma \vdash  \Delta_1, \alpha, \Delta_2 $}
	\DisplayProof
\\
&
\\

	\AxiomC{$\Gamma_1, \alpha, \beta, \Gamma_2 \vdash \Delta $}
	\RightLabel{\scriptsize (left \textbf{X})}
	\UnaryInfC{$\Gamma_1, \beta, \alpha, \Gamma_2 \vdash  \Delta $}
	\DisplayProof

&	
	%Right THEN
	\AxiomC{$\Gamma \vdash \Delta_1, \alpha, \beta, \Delta_2 $}
	\RightLabel{\scriptsize (right \textbf{X})}
	\UnaryInfC{$\Gamma \vdash \Delta_1, \beta, \alpha, \Delta_2 $}
	\DisplayProof
	
	%~ &
	%~ 
	%~ %Right THEN
	%~ \AxiomC{$\Gamma , \alpha \vdash \beta, \Delta $}
	%~ \RightLabel{\scriptsize (right $\rightarrow$)}
	%~ \UnaryInfC{$\Gamma \vdash (\alpha \rightarrow \beta), \Delta $}
	%~ \DisplayProof
	
	%~ \\
\end{tabularx}
\end{center}
\begin{prooftree}
	\AxiomC{$\Gamma , \alpha \vdash \Delta $}
	\AxiomC{$\Gamma \vdash \alpha , \Delta $}
	\RightLabel{\scriptsize (cut rule)}
	\BinaryInfC{$\Gamma \vdash \Delta $}
\end{prooftree}
\caption{Reglas estructurales.}\label{reglasEstructurales}
\end{figure}

El cálculo de secuentes para lógica de primer orden provee todas las reglas del cálculo de secuentes para proposicional junto con las reglas específicas para los cuantificadores que se muestran en la figura \ref{reglasCuantificadores}.
Las reglas (\emph{right $\forall$}) y (\emph{left $\exists$}) suelen llamarse reglas de cuantificación universal, mientras a (\emph{left $\forall$}) y (\emph{right $\exists$}) se las llama reglas de cuantificación existencial.
A las constantes frescas que aparecen en la demostración luego de la aplicación de una regla de cuantificación universal se las llama a menudo \emph{constantes de Skolem}, y se acostumbra referirse a estas aplicaciones como \emph{skolemizaciones} de las fórmulas cuantificadas.

\bigskip

\begin{figure}
\begin{tabularx}{\textwidth}{Xc Xc}
	
	%Left ForAll
	\AxiomC{$\Gamma , \alpha\{x \leftarrow t\} \vdash \Delta $}
	\RightLabel{\scriptsize (left $\forall$)}
	\UnaryInfC{$\Gamma , (\forall x:\alpha)  \vdash \Delta $}
	\DisplayProof
	
	&
	
	%Right ForAll
	\AxiomC{$\Gamma \vdash \alpha\{x \leftarrow a\} , \Delta $}
	\RightLabel{\scriptsize (right $\forall$)}
	\UnaryInfC{$\Gamma \vdash (\forall x:\alpha), \Delta $}
	\DisplayProof

	\\ & \\
	
	%Left Exist
	\AxiomC{$\Gamma , \alpha\{x \leftarrow a\} \vdash \Delta $}
	\RightLabel{\scriptsize (left $\exists$)}
	\UnaryInfC{$\Gamma , (\exists x:\alpha)  \vdash \Delta $}
	\DisplayProof
		
	&
	
	%Right Exist
	\AxiomC{$\Gamma \vdash \alpha\{x \leftarrow t\} , \Delta $}
	\RightLabel{\scriptsize (right $\exists$)}
	\UnaryInfC{$\Gamma \vdash (\exists x:\alpha), \Delta $}
	\DisplayProof
		
\end{tabularx}

\bigskip

\small{Notación: $t$ es un término, $a$ es una constante fresca, y $\alpha\{ x \leftarrow t \}$ es el resultado de sustituir las ocurrencias de $x$ como variable libre en A, con el término $t$ (de ser necesario se renombran las variables ya ligadas en $A$ para evitar capturar variables libres en $t$).}


\caption{Reglas del cálculo de secuentes para cuantificadores.} \label{reglasCuantificadores}
\end{figure}

\section{Demostraciones heterogéneas}
\label{heterogenious proofs}

\marginnote{Notar que el s'imbolo $\vdash_L$ se utiliza tanto para separar conjuntos de f'ormulas en un secuente como para representar la noci'on de ``demostrabilidad''. S'olo se va a aclarar del cu'al de los dos se trata en los casos donde no quede claro por el contexto.}[3cm]

Hoy en dia existe una gran variedad de lenguajes l'ogicos cada uno con sus caracter'isticas. Dependiendo de su poder expresivo, cada lenguaje puede describir mejor algunos aspectos del sistema pero no otros. Generalmente al elegir especificar un sistema utilizando un solo lenguaje nos topamos con este \textit{tradeoff}.

Naturalmente surge la idea de combinar varios lenguajes y usar cada uno para describir aspectos del sistema de modo de aprovechar mejor las caracter'isticas propias de cada uno. Siguiendo 'este concepto se vuelve necesario poder traducir lo escrito de un lenguaje l'ogico a otro \cite{goguen:jacm-39_1,meseguer:lc87,tarlecki:sadt-rtdts95}.

As'i se puede probar que dados los lenguajes l'ogicos $L'$ y $L$, tales que 

\begin{itemize}
\item $L'$ un lenguaje con c'alculo completo. \todomm{Mencionar más arriba qué es un lenguaje con cálculo completo. Tal vez en la introducción al capítulo, donde se puede decir que el cálculo de secuentes es un cálculo completo para ciertas lógicas (primer orden, por ejemplo). Para esto podés usar el libro que te mandé cuando empezamos a trabajar - si no lo tenés por ahí, avisame y te lo busco de nuevo.}
\item $L$ un lenguaje que se puede traducir a $L'$.
\item $T_{L \to L'}(\alpha)$ una traducci'on de $\alpha$ del lenguaje $L$ a $L'$ 
\item $T_{L \to L'}(\Gamma)$ una traducci'on de f'ormulas de $\Gamma$ del lenguaje $L$ a $L'$.
\end{itemize}
si es posible demostrar en el lenguaje $L'$ la fórmula $T_{L \to L'} (\alpha)$, entonces $\alpha$ es universalmente válida en la clase de modelos de $L$.

Con lo cu'al se obtiene que
$$\Gamma \models_L \alpha \mbox{ si y sólo si } T_{L \to L'}(\Gamma) \vdash_{L'} T_{L \to L'}(\alpha)$$
Adem'as si $L$ tambi'en posee un c'alculo no necesariamente completo, se obtiene que:
$$\mbox{Si } T_{L \to L'}(\Gamma) \vdash_{L'} T_{L \to L'}(\alpha) \mbox{ entonces } \Gamma \vdash_L \alpha$$

Como consecuencia tiene sentido extender el c'alculo dando soporte a las reglas tanto del c'alculo del lenguaje $L$ como las del c'alculo del lenguaje $L'$ agregando una regla adicional para soportar las traducciones de un lenguaje a otro:

\begin{center}
	\AxiomC{$T_{L \to L'}(\Gamma) \vdash_{L'} T_{L \to L'}(\alpha)$}
	\RightLabel{\scriptsize (translation)}
	\UnaryInfC{$\Gamma \vdash_L \alpha$}
	\DisplayProof
\end{center}

De esta forma es posible realizar demostraciones de problemas definidos en el lenguaje $L$ y luego pasar al lenguaje $L'$ aplicando la regla \textit{translation}. 'Este tipo de c'alculos fue estudiado por Borzyszkowski en \cite{borzyszkowski:tcs-286_2}.


\section{Heterogenius}
\label{section:Heterogenius}

Heterogenius \cite{heterogenius} es un demostrador de teoremas heterog'eneo que, como su principal objetivo, permite realizar demostraciones interactivas mediante el c'alculo de secuentes, representando lo que se quiere demostrar en distintos lenguajes, aplicando reglas de ese c'alculo y usando herramientas autom'aticas externas. 

Su principal caracter'istica es que soporta demostraciones heterog'eneas. En otras palabras las demostraciones pueden realizarse utilizando secuentes escritos en distintos lenguajes. Los lenguajes soportados actualmente son dos: \textit{Alloy} \cite{alloy} y \textit{PDOCFA} \cite{pdocfa}. Entre otras, Heterogenius presenta las siguientes caracter'isticas:

\begin{itemize}

\item Pone en pr'actica la idea de \textit{'arbol de an'alisis} el cu'al es el elemento principal donde se realiza el proceso de demostraci'on. 
\todoag{Acá lo que debería decir es que el árbol de análisis es un concepto que extiende a la idea de árbol de demostración (como en el cálculo de secuentes). Esto tiene que estar explicado en el paper de Manu. Si no, avisanos. - no encontre la explicacion}

\item Separa conceptualmente y arquitect'onicamente las herramientas utilizadas durante el an'alisis.

\item Ofrece interacci'on con varias herramientas externas.

\item Posee una arquitectura modular y extensible.

\item Posee una interfaz de usuario moderna que intenta ser más intuitiva que las interfaces textuales que aún son la tendencia predominante en los sistemas de demostración interactivos.

\end{itemize}

El 'arbol de an'alisis es donde se realizan todas las acciones y es donde se refleja el camino tomado para lograr una demostraci'on exitosa. Cada nodo del 'arbol representa un secuente en alg'un lenguaje soportado por Heterogenius. Las aristas se corresponden con las acciones aplicadas a cada secuente. Dependiendo del lenguaje en el que est'e el secuente se habilitan distintas acciones, pero en general se las puede dividir en tres categor'ias:

\begin{description}
\label{clasificacion}
\item[\textbf{reglas del c'alculo de secuentes}:] son acciones que transforman un secuente en otro. Algunas pueden producir múltiples secuentes (por ejemplo la acci'on \emph{Case}) creando varias ramas que tienen que ser demostradas para lograr un resultado en el nodo raiz.

\item[\textbf{traducciones}:] traducen un secuente de un lenguaje a otro. Dependiendo de la expresividad del lenguaje esta traducci'on puede preservar completamente la semántica o sólo parcialmente.

\item[\textbf{búsquedas de contraejemplos}:] implican el uso de herramientas externas para buscar contraejemplos para intentar validar el secuente donde se aplica la acci'on. 
Dependiendo del lenguaje del secuente y del poder de la herramienta seleccionada para realizar la búsqueda, estos procesos pueden verse como análisis parciales ya que podría ser imposible agotar todo el espacio de búsqueda. En tales contextos, una búsqueda infructuosa no brinda mayor información. Por el contrario, la existencia de un contraejemplo es prueba suficiente para saber que la rama en cuestión no podrá ser cerrada.
\end{description}

Debido a la extensibilidad fue uno de los conceptos que guiaron su diseño, Heterogenius sirve como  una plataforma en la cu'al se puede experimentar con lenguajes y herramientas autom'aticas nuevas.


\section{TPTP-FOF}
\label{capitulo:TPTP-FOF}

Desde el a\~no 1996 se celebra anualmente la competencia de demostradores autom'aticos \emph{CADE ATP System Comptetion} \cite{casc} como parte de la conferencia internacional sobre deducci'on autom'atica CADE.
Dicha competencia es uno de los principales eventos donde varios de los m'as famosos grupos de investigaci'on que trabajan en el desarrollo de herramientas de demostraci'on autom'atica se reunen para evaluar la \emph{performance} de sus sistemas. Los sistemas participanetes compiten analizando la mayor cantidad de problemas de un mismo conjunto en la menor cantidad de tiempo posible. Debido a 'esto, 'este evento resulta ser una muy buena prueba para evaluar el funcionamiento de las herramientas autom'aticas disponibles. Varios de los principales desarrolladores de las herramientas que se utilizaron aqui, participan regularmente de esta competencia.

Gracias a ello, un gran n'umero de herramientas de demostraci'on autom'atica aceptan entradas escritas en el lenguaje propio de la competencia: \textit{TPTP-FOF} \cite{fof}. Es por ello que hemos decidido agregar el mencionado lenguaje al motor de Heterogenius, de modo de permitir la integraci'on con herramientas actuales y abrir el camino para la interacci'on con las futuras tecnologías que soporten TPTP-FOF.

\textit{TPTP} (mil problemas para demostradores autom'aticos, por sus siglas en ingl'es) \cite{tptp} es una biblioteca de problemas para demostradores autom'aticos. La principal motivaci'on para \textit{TPTP} es permitir el testeo y evaluaci'on de diferentes sistemas de demostraci'on autom'atica de teoremas.

Los problemas est'an expresados en cuatro lenguajes: \textit{THF} (Types Higher Order Form), \textit{TFF} (Types First Order Form), \textit{FOF} (First Order Form) y \textit{CNF} (Clause Normal Form). \textit{FOF} \cite{fof} es un lenguaje de l'ogica de primer 'orden con igualdad. Una especificaci'on de \textit{TPTP-FOF} es una lista de f'ormulas que adem'as tienen un nombre y un tipo.

Por ejemplo en la fig. \ref{ejemplofof} se muestra una especificaci'on v'alida que define seis axiomas y una conjetura. La conjetura es una f'ormula de tipo \textit{conjecture} lo cu'al le indica al demostrador que es la f'ormula que se quiere probar. 

El primer axioma que se llama \textbf{john} define que la proposici'on  \textit{human()} con la constante \textit{john} es verdadera.
El axioma \textbf{all\_created\_equal} indica con el cuantificador universal que para cualquier par de variables \textit{H1} y \textit{H2} se cumple que si \textit{H1} y \textit{H2} son humanos (o sea que es verdad que \textit{human(H1)} y \textit{human(H2)} entonces se cumple que son creados de la misma forma (\textit{created\_equal(H1, H2)}).
El axioma \textbf{john\_failed} define que la proposici'on \textit{grade()} evaluada con la constante \textit{john} es igual a \textit{f} que es otra constante.
El siguiente exioma introduce un cuantificador existencial con lo cu'al se indica que existe una variable \textit{H} tal que \textit{human(H)} y \textit{grade(H)=a} valen.
Los siguientes axiomas definen que \textit{a} es distinto del \textit{f} y que no se pueden componer \textit{human} con \textit{grade}.

Lo definido por los axiomas se puede interpretar como: (1) \textit{john} es humano; (2) todos los humanos son creados de la misma forma; (3) la calificaci'on de \textit{john} es \textit{f}; (4) alguien tiene la calificaci'on \textit{a}; (5) \textit{a} es distinta de \textit{f} (6) las calificaciones no son humanos.

A su vez la conjetura pregunta si existe un humano que no es \textit{john}.


\begin{figure}[]
\makebox[\textwidth]{}{
\small
\begin{verbatim}
fof(john,axiom,( human(john) )).

fof(all_created_equal,axiom,( 
    ![H1,H2] : ( ( human(H1) & human(H2) )
	    => created_equal(H1,H2) ) )). 

fof(john_failed,axiom,( grade(john) = f )). 

fof(someone_got_an_a,axiom,( ?[H] : ( human(H) & grade(H) = a ) )). 

fof(distinct_grades,axiom,( a != f )). 

fof(grades_not_human,axiom,( ![G] : ~ human(grade(G)) )). 

fof(someone_not_john,conjecture,( ?[H] : ( human(H) & H != john ) )). 
\end{verbatim}
\normalsize}
\caption{Especificaci'on t'ipica en el lenguaje \textit{TPTP-FOF}. La sintaxis contiene todos los elementos de la l'ogica de primer 'orden. \textbf{![X]} es el cuantificador universal sobre la variable \textit{X}; \textbf{?[X]} es el cuantificador existencial sobre la variable \textit{X}. \textit{$human()$}, \textit{$grade()$}, \textit{$created\_equal()$} son predicados; las variables empiezan con una letra mayuscula y las constantes con minuscula. Los operadores disponibles son $\sim$, $|$, $\&$, $=>$, $<=>$, $=$ y $!=$ y corresponden con la negaci'on, la disyunci'on, la conjunci'on, la implicaci'on, la doble implicaci'on, la igualdad y la desigualdad.}
\label{ejemplofof}
\end{figure}


\section{Herramientas autom'aticas utilizadas}

En 'esta secci'on vamos a explicar somaramente el funcionamiento de las herramientas autom'aticas con las que extendimos la funcionalidad de \textit{Heterogenius}.

\subsection{E-Prover}

E-Prover es un demostrador de teoremas de l'ogica de primer orden con igualdad. El demostrador acepta como entrada una lista de cl'ausulas o f'ormulas de l'ogica de primer orden con una conjetura. El sistema intenta encontrar una demostraci'on para la conjetura asumiendo los axiomas especificados.

Ya que E-Prover adem'as permite generar modelos se lo puede utilizar tambi'en como un buscador de contraejemplos.

Internamente la principal t'ecnica usada es el c'alculo por superposici'on. Es un c'alculo que permite razonar en l'ogica de primer orden. Se ha desarrollado a principios de 1990 y combina los principios de resoluci'on de primer orden con el manejo de igualdad basada en orden desarrollado en el contexto del algoritmo de completitud de \textit{Knuth-Bendix}\cite{knuthbendix}. Como la mayor'ia de c'alculos de l'ogica de primer orden, superposici'on trata de mostrar la insatisfacibilidad de un conjunto de cl'ausulas realizando demostraciones por refutaci'on. Superposici'on es completo para la refutaci'on, o sea dada una cantidad ilimitada de recursos y una estrategia de derivaci'on justa, se va a llegar a una contradicci'on empezando por cualquier conjunto de clausulas insatisfacible.

El demostrador opera en cuatro fases distintas:
\begin{itemize}
\item Clausificaci'on: se transforman las f'ormulas de entrada en f'orma normal clausal de tal manera que la f'ormula resultante es insatisfacible si y s'olo si el problema original es demostrable. Se usa una variaci'on del algoritmo detallado en \cite{smallcnf2001}.

\item Una etapa opcional de simplificaci'on de la forma clausal del problema.

\item An'alisis del problema y selecci'on de par'ametros 'optimos para la b'usqueda de una demostraci'on.

\item La etapa de saturaci'on que se implementa sobre un algoritmo del c'alculo de superposici'on descripto en \cite{bg94} y \cite{aicom2002} que trata de encontrar la cl'ausula vac'ia, lo que permite determinar que la cl'ausula original es insatisfacible y, por lo tanto, el problema de entrada es demostrable. E-Prover implementa el algoritmo de superposici'on con selecci'on del literal negativo y con factorizaci'on de igualdad. Se incorporan varias t'ecnicas de simplificaci'on como por ejemplo reescritura incondicional, subsunci'on, eliminaci'on de tautologias semanticas y corte de literales contextuales. Tambi'en se implementa la eliminaci'on de redundancia AC documentada en \cite{jsc2003} y la introducci'on controlada de definiciones para partes independientes de cl'ausulas \cite{ijcai2001}.
\end{itemize}

\subsubsection{B'usqueda de demostraci'on}

E-Prover utiliza una variante del algoritmo de ``cl'ausula dada'' \cite{dk97}. El estado de la demostraci'on se representa por dos conjuntos de cl'ausulas: el conjunto \textbf{P} de cl'ausulas procesadas (inicialmente vac'io) y el conjunto \textbf{U} de cl'ausulas sin procesar. Las cl'ausulas en \textbf{U} se ordenan de acuerdo a una funci'on de evaluaci'on he'uristica y se procesan en ese orden.


%El procesamiento primero simplifica la cl'ausula seleccionada \textit{g} con todas las cl'ausulas en \textbf{P}, luego simplifica \textbf{P} con \textit{g}, moviendo todas las cl'ausulas afectadas desde \textbf{P} a \textbf{U}. Luego se computan todas las consecuencias directas entre \textit{g} y \textbf{P} que pueden ser derivadas utilizando las reglas de inferencia (superposici'on, factoreo de igualdad y resoluci'on de igualdad). Las nuevas cl'ausulas son simplificadas respecto a \textbf{P} y son agregadas a \textbf{U}, a su vez \textit{g} es agregado a \textbf{P}.

\begin{figure}
\makebox[\textwidth]{}{
\small
\begin{lstlisting}[mathescape=true,escapeinside={@}{@}]

@\textbf{U}@: clausulas sin procesar. Inicialmente clausulas
    de la especificacion.
@\textbf{P}@: clausulas procesadas. Inicialmente vacio.

mientras @\textbf{U}@ $\neq$ {}
  @\textbf{g}@ = eliminar(@\textbf{U}@)
  @\textbf{g}@ = simplificar(@\textbf{g}@, @\textbf{P}@)
  si @\textbf{g}@ == vacia:
    Se encontro una demostracion.
  
  si @\textbf{g}@ no es redundante respecto a @\textbf{P}@:
    @\textbf{T}@ = {@\textbf{c}@ $\in$ @\textbf{P}@ | @\textbf{c}@ redundante o simplificable respecto a @\textbf{g}@}
    @\textbf{P}@ = (@\textbf{P}@ \ @\textbf{T}@) $\cup$ {@\textbf{g}@}
    @\textbf{T}@ = @\textbf{T}@ $\cup$ generar(@\textbf{g}@, @\textbf{P}@)
    para cada @\textbf{c}@ $\in$ @\textbf{T}@:
      @\textbf{c}@ = simplificar(@\textbf{c}@, @\textbf{P}@)
      si @\textbf{c}@ no es trivial:
        @\textbf{U}@ = @\textbf{U}@ $\cup$ {@\textbf{c}@}
Exito, el conjunto @\textbf{U}@ original es satisfacible.

\end{lstlisting}
\normalsize}
\caption{Pseudoc'odigo del procedimiento de b'usqueda usado por \texttt{E-Prover}}
\label{fig:codigo_clausula_dada_eprover}
\end{figure}

El procedimiento de b'usqueda de demostraci'on mostrado en la Fig. \ref{fig:codigo_clausula_dada_eprover} es diferente del algoritmo de ``cl'ausula dada'' implementado por \texttt{Otter}\cite{otter} y otros demostradores que tambi'en utilizan el conjunto \textbf{U} para la simplificaci'on. 'Esta variante tambi'en se utiliza en \texttt{Waldmeister}\cite{BH96} y forma parte adem'as del algoritmo principal de \texttt{Vampire} \cite{vampire} y \texttt{SPASS}\cite{spass}.

\subsubsection{Heur'isticas}

E-Prover usa una serie de heur'isticas para: (1) la elecci'on del ordenamiento de t'erminos que restringe las inferencias y controla la reescritura; (2) la elecci'on de la siguiente cl'ausula de \textbf{U} y opcionalmente (3) la elecci'on de literales para la inferencia en las cl'ausulas que tienen por lo menos un literal negativo.

La selecci'on de las cl'ausulas dadas es controlada por varias colas de prioridad de las cuales se seleccionan las cl'ausulas en un esquema ``round robin pesado''. Cada cola es ordenada de acuerdo a una funci'on de prioridad fija y una funci'on de evaluaci'on de cl'ausula parametrizada. Las funciones de evaluaci'on de cl'ausulas pueden ser orientadas a objetivos, pueden estar enfocadas en algunos s'imbolos en particular y pueden tomar en cuenta el ordenamiento de t'erminos. 


La selecci'on de literales es controlada por una de las cien heur'isticas de selecci'on que est'an codificadas en E-Prover.



\subsection{SPASS}

SPASS es un demostrador autom'atico de teoremas basado en saturaci'on para l'ogica de primer orden con igualdad. Como se menciona en la descripci'on oficial, SPASS es 'unico debido a que combina el c'alculo de superposici'on con reglas espec'ificas de inferencia/reducci'on para tipos y una regla de partici'on para an'alisis de casos motivada por la regla $\beta$ del m'etodo de \textit{tableau} anal'itico y el an'alisis de casos empleado en el procedimiento \textit{Davis-Putnam}. El 'enfasis de SPASS est'a puesto en las diferentes t'ecnicas de inferencia, reducci'on, simplificaci'on, y no as'i en las heur'isticas.

Todo el funcionamiento interno de SPASS est'a basado en l'ogica de primer 'orden y consiste en reglas de inferencia que generan nuevas cl'ausulas y reglas de reducci'on que reducen la cantidad de cl'ausulas o las simplifican. Hay una gran variedad de reglas de inferencia y reducci'on para conjuntos de cl'ausulas que pueden ser combinadas para ajustarse a varios c'alculos de primer orden completos y correctos. La particularidad de SPASS es que se introduce una regla nueva que es la regla de partici'on que soporta un an'alisis de casos expl'icito. De 'esta manera la regla de partici'on introduce una segunda dimensi'on en el m'etodo de demostraci'on de teoremas basado en saturaci'on.

La tercera dimensi'on que se considera en SPASS son las restricciones, informaci'on extra que se adjunta a las cl'ausulas restrigiendo su sem'antica y su uso respecto al c'alculo. Algunas de las restriciones son \textit{de orden}, forzando que los t'erminos substituidos satisfagan las funciones de restricci'on adjuntas y restricciones \textit{de tipo} que garantizan que la instanciaci'on de las variables corresponda al tipo especificado.


\subsection{Mace4}

Mace4 es un buscador de modelos finitos basado en \textit{backtracking} y puede ser considerado una variaci'on del algor'itmo \textit{DPLL/Davis-Putnam-Logemann-Loveland} \cite{dpll}. La principal diferencia radica en la selecci'on de celdas en la matriz que representa el modelo buscado, la selecci'on de valores a asignar y la propagaci'on de las asignaciones \cite{mace4_manual}.

\subsubsection{Inicializaci'on}

Mace4 siempre hace la b'usqueda sobre un modelo de dominio finito. Incluso cuando no se especifica un tamaño de dominio Mace4 itera sobre los posibles tamaños ejecutando la rutina de b'usqueda sobre cada dominio candidato.

La inicializaci'on comienza con la creaci'on de matrices para las funciones. Para cada funci'on de aridad $k$ se crea una matriz $n^k$ donde $n$ es el tamaño del dominio.

\subsubsection{B'usqueda}

La b'usqueda se realiza con una recursi'on backtracking y su pseudoc'odigo es:

\begin{figure}[htbp]
\makebox[\textwidth]{}{
\small
\begin{verbatim}
procedure search:
  cell = select_cell()
  top = last_value_to_consider()
  foreach i (0 ... top)
    if (assign_and_propagate(cell,i)):
      search()
      undo_assignments();
\end{verbatim}
\normalsize}
\caption{Algor'itmo de b'usqueda utilizado en \textit{Mace4}.}
\label{fig:mace4_busqueda}
\end{figure}

Las partes m'as importantes del algor'itmo son las heuristicas de selecci'on de celdas; determinaci'on de los valores a asignar a la celda seleccionada; y propagaci'on de la asignacion.

\paragraph{Selecci'on de celdas}

La selecci'on de celdas se divide en dos fases. La primera determina el conjunto de celdas candidatas. La segunda fase elige una celdas del conjunto de candidatas previamente preparadas.

\begin{itemize}

\item{Armado del conjunto de celdas candidatas}: Existen tres heur'isticas para el armado del conjunto: \textit{Orden lineal}, se seleccionan todas las celdas abiertas; \textit{Orden concentrico}, sea $i$ el 'indice m'aximo m'as chico \todoag{no entendí lo del “índice máximo más chico”. ¿Lo podrías explicar con más detalle - yo tampoco, se explica asi en el manual: https://goo.gl/ZLVbZp”} de una celda abierta, se seleccionan todas las celdas con el 'indice $i$ m'aximo; \textit{Orden de banda conc'entrica}, todas las celdas con el 'indice m'aximo menor o igual al valor actual de restricci'on son candidatas. Si no se encuentra ninguna se revierte al orden conc'entrico.

\item{Selecci'on de la celda}. Se utilizan cuatro m'etodos:

\begin{itemize}
\item{m'etodo 0}: Se selecciona la primer celda candidata.
\item{m'etodo 1}: Se selecciona la candidata con el mayor n'umero de ocurrencias en el actual conjunto de cl'ausulas.
\item{m'etodo 2}: Se selecciona la candidata que podr'ia producir el mayor n'umero de propagaciones. La motivaci'on est'a en llenar la tabla lo antes posible.
\item{m'etodo 3}: Se selecciona la candidata que podr'ia causar el mayor n'umero de contradicciones. La motivaci'on es cortar la mayor cantidad de caminos lo antes posible.
\item{m'etodo 4}: Se selecciona la candidata con la menor cantidad de valores posibles.
\end{itemize}

\end{itemize}


\paragraph{Asignaci'on de la celda}
Una vez que se seleccionan las celdas abiertas, se debe determinar el conjunto de valores a considerar para la asignaci'on. Si se trata de una celda booleana siempre se prueba con $0$ y $1$. Si no, se prueba con todos los valores del dominio $[0,\dots,n-1]$ si no se est'a usando la heur'istica \textit{LNH} (Least Number Heuristic). Esta heur'istica fue usada por primera vez en \cite{zhang95} y \cite{zhang96} y su objetivo es tratar de eliminar alg'un isomorfismo en la b'usqueda. La idea principal es de mantener el dominio particionado en valores asimetricos y valores sim''etricos. S'olo uno de los valores sim'etricos debe ser probado. En la pr'actica esta heur'istica es efectiva en los niveles m'as altos del 'arbol de b'usqueda.


\paragraph{Propagaci'on}

\begin{itemize}

\item{Propagaci'on Positiva}:
La propagaci'on positiva se aplica siempre al tener una asignaci'on. Por ejemplo si $f(2,3)=4$, entonces todas las ocurrencias de $f(2,3)$ pueden ser reemplazadas por $4$. Cada uno de estos reemplazos puede producir otros reemplazos en cadena. Por ejemplo $g(f(2,3))=5$ al ser reemplazado una vez termina siendo $g(4)=5$, pero esto genera una nueva asignaci'on y reemplazos posibles.

\item{Propagaci'on Negativa}:
La propagaci'on negativa tiene como objetivo reducir la cantidad de posibles valores de una asignaci'on hasta tener un solo valor que se pueda asignar. Por ejemplo sabiendo que $f(2,3)=4$ y $f(2, g(5))\neq4$ se puede inferir que $g(5)\neq3$. Los cuatro tipos de clausulas que pueden producir propagaciones negativas son:

\medskip
\begin{center}
\begin{tabular}{ |l|l| }
\hline
Tipo & Ejemplo \\
\hline
Asignaci'on & $f(2,3)=4$ \\
Casi asignaci'on & $f(2,g(5))=4$ \\
Eliminaci'on & $f(2,3)\neq4$ \\
Casi eliminaci'on & $f(2,g(5))\neq4$ \\
\hline
\end{tabular}
\end{center}
\medskip

La asignaci'on se empareja con casi eliminaci'on y la eliminaci'on con casi asignaci'on. Cualquiera de estas cuatro cl'ausulas puede producir una propagaci'on negat'iva

\end{itemize}
