\chapter{Preliminares}

\section{Cálculo de secuentes} \label{calculo-secuentes}

El cálculo de secuentes es una familia de sistemas de demostración. Los primeros cálculos de secuentes (los sistemas LK y LJ) fueron desarrollados originalmente por Gerhard Gentzen en la década del `30\cite{gentzen1935}.

En el cálculo de secuentes una demostración se construye como un árbol en el que sus nodos son secuentes.
Un secuente es un par ordenado de listas finitas de fórmulas, y lo notamos $ \Gamma \vdash  \Delta$.
Su interpretación intuitiva puede pensarse de esta manera: un secuente $\Gamma \vdash  \Delta$ es válido en un modelo semántico \sem{M}, si $\textrm{\sem{M}} \vDash (\bigwedge_{\gamma \in \Gamma} \gamma \Rightarrow \bigvee_{\delta \in \Delta} \delta )$.
Luego, un secuente es \emph{universalmente válido} (o simplemente \emph{válido}) si es válido para todo modelo semántico.
Correspondientemente, un secuente $\Gamma \vdash  \Delta$ es \emph{inválido} si existe un modelo semántico \sem{N} tal que $\textrm{\sem{N}} \vDash (\bigwedge_{\gamma \in \Gamma} \gamma \wedge \bigwedge_{\delta \in \Delta} \neg\delta )$.
En tal caso, diremos que \sem{N} es un \emph{contraejemplo} para ese secuente.

Para demostrar la fórmula $\phi$, se parte de un árbol con un único nodo: $ \vdash \phi $ (un abuso de notación para el secuente $ \emptyset \vdash \{\phi\} $). Mediante la aplicación de reglas de inferencia sobre las hojas, el árbol crece en altura incorporando nuevos secuentes como hijos del nodo sobre el cual se aplica la regla.
La demostración se considera terminada cuando todas las hojas del árbol son de la forma $\alpha \vdash \alpha$ con $\alpha$ una fórmula, los cuales son llamados \emph{secuentes triviales} o \emph{secuentes terminales}.
%con $ \Sigma \cap \Gamma = \emptyset $ 

Las reglas de inferencia tienen la forma que se muestra a continuación,
\begin{prooftree}
	\AxiomC{$\Gamma_{1} \vdash \Delta_{1} $}
	\AxiomC{$\dots$}
	\AxiomC{$\Gamma_{n} \vdash \Delta_{n} $}
	\RightLabel{\scriptsize ($R$)}
	\TrinaryInfC{$\Gamma \vdash \Delta $}
\end{prooftree}
y se pueden interpretar de esta manera: dada una hoja del árbol de demostración de la forma $\Gamma \vdash \Delta$, la aplicación de la regla $R$ modifica el árbol agregando $n$ nuevas hojas de la forma $\Gamma_{i}\vdash\Delta_{i}$ como hijas del secuente $\Gamma \vdash \Delta$\footnote{Si bien las reglas pueden interpretarse en sentido inverso, preferimos esta dirección debido a que vamos a darle más atención al proceso de construcción de la demostración que a su lectura posterior.}.

\bigskip

A modo de ejemplo, veamos cómo construir la demostración de la fórmula proposicional $(p \vee \neg p) \wedge (q \rightarrow q)$ utilizando un cálculo de secuentes para esa lógica.
La fórmula dada es una conjunción de tautologías, ya que $(p \vee \neg p)$ y $(q \rightarrow q)$ son válidas para toda valuación, y por lo tanto es ella misma una tautología.
Luego, por la compleción del cáculo, existe una demostración como las descriptas más arriba para esta fórmula.


En la figura \ref{reglasProposicional} pueden verse algunas reglas del cálculo que nos serán útiles en la demostración.
%~ \bigskip

\begin{figure}
\begin{tabularx}{\textwidth}{Xc Xc}
	
	%Right OR
	\AxiomC{$\Gamma \vdash \alpha, \beta, \Delta $}
	\RightLabel{\scriptsize (right $\vee$)}
	\UnaryInfC{$\Gamma \vdash (\alpha \vee \beta), \Delta $}
	\DisplayProof
	
	&
	
	%Right THEN
	\AxiomC{$\Gamma , \alpha \vdash \beta, \Delta $}
	\RightLabel{\scriptsize (right $\rightarrow$)}
	\UnaryInfC{$\Gamma \vdash (\alpha \rightarrow \beta), \Delta $}
	\DisplayProof
	
	\\ & \\
	
	%Right NOT
	\AxiomC{$\Gamma , \alpha \vdash \Delta $}
	\RightLabel{\scriptsize (right $\neg$)}
	\UnaryInfC{$\Gamma \vdash (\neg\alpha) , \Delta $}
	\DisplayProof
	
	&
	
	%Right AND
	\AxiomC{$\Gamma \vdash \alpha, \Delta $}
	\AxiomC{$\Gamma \vdash \beta, \Delta $}
	\RightLabel{\scriptsize (right $\wedge$)}
	\BinaryInfC{$\Gamma \vdash (\alpha \wedge \beta) , \Delta $}
	\DisplayProof
	
\end{tabularx}
\caption{Algunas reglas del cálculo de secuentes para lógica proposicional.}\label{reglasProposicional}
\end{figure}
%~ \bigskip

Inicialmente tenemos, como raíz del árbol de demostración, al secuente que no tiene antecedentes y sólo tiene como consecuente la fórmula que queremos demostrar: $$ \vdash (p \vee \neg p) \wedge (q \rightarrow q)$$
Aplicando la regla (\textit{right} $\wedge$) se obtienen dos secuentes hijos, los cuales indican nuestras nuevas obligaciones de demostración:
$$  \vdash (p \vee \neg p) \;\;\;\;\;\;\; \vdash (q \rightarrow q) $$
Ataquemos primero el segundo secuente recién obtenido. Al aplicar la regla (\textit{right} $\rightarrow$), se obtiene un único secuente hijo: $$ q \vdash q $$
Como este es un secuente trivial, ya podemos considerar cerrada esta rama de la demostración.
Volvamos entonces al secuente $\vdash (p \vee \neg p)$. Aplicando la regla (\textit{right} $\vee$), obtenemos un único secuente hijo:
$$ \vdash p \; , \; \neg p $$
Finalmente aplicando la regla (\textit{right} $\neg$) obtenemos nuevamente un único secuente hijo $$ p \vdash p $$
Dado que es un secuente trivial, también podemos considerar cerrada esta rama de la demostración.
Así llegamos a obtener un árbol con raíz $ (p \vee \neg p) \wedge (q \rightarrow q)$ y cuyas hojas son todas secuentes triviales.
Luego, hemos demostrado dicha fórmula.
En la figura \ref{arbolDemoEjemplo} se muestra la demostración anterior escrita utilizando la notación tradicional del cálculo de secuentes.

\begin{figure}
%\missingfigure{Arbol de demostración tradicional}

\begin{prooftree}
	\AxiomC{$p \vdash p$}
	\RightLabel{\scriptsize(right $\neg$)}	
		\UnaryInfC{$\vdash p , \neg p$}
	\RightLabel{\scriptsize(right $\vee$)}	
		\UnaryInfC{$\vdash p \vee \neg p$}

	\AxiomC{$q \vdash q $}	
	\RightLabel{\scriptsize(right $\rightarrow$)}	
	\UnaryInfC{$\vdash (q \rightarrow q)$}

\RightLabel{\scriptsize(right $\wedge$)}
\BinaryInfC{$\vdash (p \vee \neg p) \wedge (q \rightarrow q)$}
\end{prooftree}
\caption{Ejemplo de demostración en notación tradicional}\label{arbolDemoEjemplo}
\end{figure}


%\begin{prooftree}
%\AxiomC{$P$}
%\RightLabel{\scriptsize(1)}
%\UnaryInfC{$\neg P$}
%
%%             \BinaryInfC{$\bot$}
%%             \RightLabel{\scriptsize(1)}
%%             \UnaryInfC{$\neg\neg P$}
%\end{prooftree}

\bigskip

%\mtodo{}{Este párrafo debería ir después del ejemplo; como está ahora, la frase "A modo de ejemplo" del párrafo siguiente parece referirse a la compleción y corrección del cálculo.}
%\todo{Ya lo moví, pero lo sacaría la verdad}
Para las lógicas más tradicionales (lógica proposicional, lógica de primer orden, etc.) puede demostrarse que este cáculo es completo y correcto (ver \cite{bookFoundations}, por ejemplo), es decir que para toda fórmula semánticamente válida puede construirse una demostración y que toda fórmula para la que pueda construirse una demostración es semánticamente válida.

\bigskip

Vale aclarar que, aunque se pueden hallar distintas versiones de estos cálculos aún para una misma lógica, cualquier cálculo de secuentes para lógica proposicional debe poseer más reglas que las utilizadas en el ejemplo anterior.
Por lo menos debe proveer las correspondientes a los mismos símbolos lógicos que aparecen en la figura \ref{reglasProposicional} pero para el lado izquierdo del $\vdash$.
Si bien con esas reglas puede demostrarse la corrección y compleción del cálculo (ver \cite{bookFoundations}, por ejemplo) algunos autores también incluyen otras reglas que aunque supérfluas desde el punto de vista teórico, son muy útiles al utilizar el cálculo para construir demostraciones.
Estas reglas tienen que ver con la composición estructural de los secuentes.
En la figura \ref{reglasEstructurales}, pueden verse algunos ejemplos de estas reglas: (\emph{left} \textbf{C}) y (\emph{right} \textbf{C}) permiten duplicar reglas del antecedente y del consecuente, (\emph{left} \textbf{X}) y (\emph{right} \textbf{X}) permiten cambiar el orden de las fórmulas, y la \textit{cut rule} puede interpretarse como la introducción de una nueva hipótesis a la demostración o como una separación en casos basada en el cumplimiento de la fórmula $\alpha$.

\begin{figure}
\begin{center}
\begin{tabularx}{0.7\textwidth}{Xc Xc}

	\AxiomC{$\Gamma_1, \alpha, \alpha, \Gamma_2 \vdash \Delta $}
	\RightLabel{\scriptsize (left \textbf{C})}
	\UnaryInfC{$\Gamma_1, \alpha, \Gamma_2 \vdash  \Delta $}
	\DisplayProof
	
	&

	\AxiomC{$\Gamma \vdash \Delta_1, \alpha, \alpha, \Delta_2 $}
	\RightLabel{\scriptsize (right \textbf{C})}
	\UnaryInfC{$\Gamma \vdash  \Delta_1, \alpha, \Delta_2 $}
	\DisplayProof
\\
&
\\

	\AxiomC{$\Gamma_1, \alpha, \beta, \Gamma_2 \vdash \Delta $}
	\RightLabel{\scriptsize (left \textbf{X})}
	\UnaryInfC{$\Gamma_1, \beta, \alpha, \Gamma_2 \vdash  \Delta $}
	\DisplayProof

&	
	%Right THEN
	\AxiomC{$\Gamma \vdash \Delta_1, \alpha, \beta, \Delta_2 $}
	\RightLabel{\scriptsize (right \textbf{X})}
	\UnaryInfC{$\Gamma \vdash \Delta_1, \beta, \alpha, \Delta_2 $}
	\DisplayProof
	
	%~ &
	%~ 
	%~ %Right THEN
	%~ \AxiomC{$\Gamma , \alpha \vdash \beta, \Delta $}
	%~ \RightLabel{\scriptsize (right $\rightarrow$)}
	%~ \UnaryInfC{$\Gamma \vdash (\alpha \rightarrow \beta), \Delta $}
	%~ \DisplayProof
	
	%~ \\
\end{tabularx}
\end{center}
\begin{prooftree}
	\AxiomC{$\Gamma , \alpha \vdash \Delta $}
	\AxiomC{$\Gamma \vdash \alpha , \Delta $}
	\RightLabel{\scriptsize (cut rule)}
	\BinaryInfC{$\Gamma \vdash \Delta $}
\end{prooftree}
\caption{Reglas estructurales.}\label{reglasEstructurales}
\end{figure}

El cálculo de secuentes para lógica de primer orden provee todas las reglas del cálculo de secuentes para proposicional junto con las reglas específicas para los cuantificadores que se muestran en la figura \ref{reglasCuantificadores}.
Las reglas (\emph{right $\forall$}) y (\emph{left $\exists$}) suelen llamarse reglas de cuantificación universal, mientras a (\emph{left $\forall$}) y (\emph{right $\exists$}) se las llama reglas de cuantificación existencial.
A las constantes frescas que aparecen en la demostración luego de la aplicación de una regla de cuantificación universal se las llama a menudo \emph{constantes de Skolem}, y se acostumbra referirse a estas aplicaciones como \emph{skolemizaciones} de las fórmulas cuantificadas.

\bigskip

\begin{figure}
\begin{tabularx}{\textwidth}{Xc Xc}
	
	%Left ForAll
	\AxiomC{$\Gamma , \alpha\{x \leftarrow t\} \vdash \Delta $}
	\RightLabel{\scriptsize (left $\forall$)}
	\UnaryInfC{$\Gamma , (\forall x:\alpha)  \vdash \Delta $}
	\DisplayProof
	
	&
	
	%Right ForAll
	\AxiomC{$\Gamma \vdash \alpha\{x \leftarrow a\} , \Delta $}
	\RightLabel{\scriptsize (right $\forall$)}
	\UnaryInfC{$\Gamma \vdash (\forall x:\alpha), \Delta $}
	\DisplayProof

	\\ & \\
	
	%Left Exist
	\AxiomC{$\Gamma , \alpha\{x \leftarrow a\} \vdash \Delta $}
	\RightLabel{\scriptsize (left $\exists$)}
	\UnaryInfC{$\Gamma , (\exists x:\alpha)  \vdash \Delta $}
	\DisplayProof
		
	&
	
	%Right Exist
	\AxiomC{$\Gamma \vdash \alpha\{x \leftarrow t\} , \Delta $}
	\RightLabel{\scriptsize (right $\exists$)}
	\UnaryInfC{$\Gamma \vdash (\exists x:\alpha), \Delta $}
	\DisplayProof
		
\end{tabularx}

\bigskip

\small{Notación: $t$ es un término, $a$ es una constante fresca, y $\alpha\{ x \leftarrow t \}$ es el resultado de sustituir las ocurrencias de $x$ como variable libre en A, con el término $t$ (de ser necesario se renombran las variables ya ligadas en $A$ para evitar capturar variables libres en $t$).}


\caption{Reglas del cálculo de secuentes para cuantificadores.} \label{reglasCuantificadores}
\end{figure}

\section{Demostraciones heterogéneas}
\label{heterogenious proofs}

\marginnote{Notar que el s'imbolo $\vdash_L$ se utiliza tanto para separar conjuntos de f'ormulas en un secuente como para representar la noci'on de ``demostrabilidad''. S'olo se va a aclarar del cu'al de los dos se trata en los casos donde no quede claro por el contexto.}[3cm]

Hoy en dia existe una gran variedad de lenguajes l'ogicos cada uno con sus caracter'isticas. Dependiendo de su poder expresivo, cada lenguaje puede describir mejor algunos aspectos del sistema pero no otros. Generalmente al elegir especificar un sistema utilizando un solo lenguaje nos topamos con este \textit{tradeoff}.

Naturalmente surge la idea de combinar varios lenguajes y usar cada uno para describir aspectos del sistema de modo de aprovechar mejor las caracter'isticas propias de cada uno. Siguiendo 'este concepto se vuelve necesario poder traducir lo escrito de un lenguaje l'ogico a otro \cite{goguen:jacm-39_1,meseguer:lc87,tarlecki:sadt-rtdts95}.

As'i se puede probar que dados los lenguajes l'ogicos $L'$ y $L$, tales que 

\begin{itemize}
\item $L'$ un lenguaje con c'alculo completo. \todomm{Mencionar más arriba qué es un lenguaje con cálculo completo. Tal vez en la introducción al capítulo, donde se puede decir que el cálculo de secuentes es un cálculo completo para ciertas lógicas (primer orden, por ejemplo). Para esto podés usar el libro que te mandé cuando empezamos a trabajar - si no lo tenés por ahí, avisame y te lo busco de nuevo.}
\item $L$ un lenguaje que se puede traducir a $L'$.
\item $T_{L \to L'}(\alpha)$ una traducci'on de $\alpha$ del lenguaje $L$ a $L'$ 
\item $T_{L \to L'}(\Gamma)$ una traducci'on de f'ormulas de $\Gamma$ del lenguaje $L$ a $L'$.
\end{itemize}
si es posible demostrar en el lenguaje $L'$ la fórmula $T_{L \to L'} (\alpha)$, entonces $\alpha$ es universalmente válida en la clase de modelos de $L$.

Con lo cu'al se obtiene que
$$\Gamma \models_L \alpha \mbox{ si y sólo si } T_{L \to L'}(\Gamma) \vdash_{L'} T_{L \to L'}(\alpha)$$
Adem'as si $L$ tambi'en posee un c'alculo no necesariamente completo, se obtiene que:
$$\mbox{Si } T_{L \to L'}(\Gamma) \vdash_{L'} T_{L \to L'}(\alpha) \mbox{ entonces } \Gamma \vdash_L \alpha$$

Como consecuencia tiene sentido extender el c'alculo dando soporte a las reglas tanto del c'alculo del lenguaje $L$ como las del c'alculo del lenguaje $L'$ agregando una regla adicional para soportar las traducciones de un lenguaje a otro:

\begin{center}
	\AxiomC{$T_{L \to L'}(\Gamma) \vdash_{L'} T_{L \to L'}(\alpha)$}
	\RightLabel{\scriptsize (translation)}
	\UnaryInfC{$\Gamma \vdash_L \alpha$}
	\DisplayProof
\end{center}

De esta forma es posible realizar demostraciones de problemas definidos en el lenguaje $L$ y luego pasar al lenguaje $L'$ aplicando la regla \textit{translation}. 'Este tipo de c'alculos fue estudiado por Borzyszkowski en \cite{borzyszkowski:tcs-286_2}.


\section{Heterogenius}

Heterogenius \cite{heterogenius} es un demostrador de teoremas heterog'eneo que, como su principal objetivo, permite realizar demostraciones interactivas mediante el c'alculo de secuentes, representando lo que se quiere demostrar en distintos lenguajes, aplicando reglas de c'alculo de secuentes y usando las herramientas autom'aticas externas. 

Su principal caracteristica es que soporta demostraciones heterogeneas, o sea las demostraciones pueden aceptar secuentes escritos en distintos lenguajes. Los lenguajes soportados actualmente son dos: \textit{Alloy} y \textit{PDOCFA}. Entre otras cosas Heterogenius presenta las siguientes caracteristicas:

\begin{itemize}

\item Pone en pr'actica la idea de \textit{'arbol de an'alisis} el cu'al es el elemento principal donde se realiza el proceso de demostraci'on.

\item Separa conceptualmente y arquitect'onicamente las herramientas utilizadas durante el an'alisis.

\item Ofrece interacci'on entre varias herramientas externas.

\item Posee una arquitectura modular y extensible.

\item Posee una interfaz de usuario moderna e intuitiva.

\end{itemize}

Adem'as de ser un demostrador interactivo, Heterogenius tambi'en es una plataforma en la cu'al se puede experimentar con lenguajes y herramientas autom'aticas nuevas.


\section{TPTP-FOF}

\textit{TPTP} (mil problemas para demostradores autom'aticos, por sus siglas en ingles) \cite{tptp} es una biblioteca de problemas para demostradores autom'aticos de teoremas. La principal motivaci'on para \textit{TPTP} es permitir el testeo y evaluaci'on de diferentes sistemas de demostradores autom'aticos de teoremas. Los problemas est'an en cuatro lenguajes: \textit{THF}, \textit{TFF}, \textit{FOF} y \textit{CNF}.

Anualmente se realiza la competencia \textit{CASC} (CADE ATP System Competition \cite{casc}), una competencia de demostradores autom'aticos de teoremas donde los sistemas participanetes compiten para probar la mayor cantidad de problemas de \textit{TPTP}. 'Esta competencia resulta una muy buena prueba para evaluar el funcionamiento de las herramientas autom'aticas disponibles.

\textit{FOF} \cite{fof} es un lenguaje de l'ogica de primer 'orden con igualdad. Su elecci'on de \textit{FOF} est'a motivada  por su difundido uso y soporte que tiene de las herramientas autom'aticas y su expresividad a nivel de sintaxis. Una especificaci'on de \textit{TPTP-FOF} es una lista de f'ormulas que adem'as tienen un nombre y un tipo. Por ejemplo:

\begin{verbatim}
fof(john,axiom,( 
    human(john) )).

fof(all_created_equal,axiom,( 
    ! [H1,H2] : 
      ( ( human(H1) 
         & human(H2) ) 
     => created_equal(H1,H2) ) )). 

fof(john_failed,axiom,( 
    grade(john) = f )). 

fof(someone_got_an_a,axiom,( 
    ? [H] : 
      ( human(H) 
      & grade(H) = a ) )). 

fof(distinct_grades,axiom,( 
    a != f )). 

fof(grades_not_human,axiom,( 
    ! [G] : ~ human(grade(G)) )). 

fof(someone_not_john,conjecture,( 
    ? [H] : 
      ( human(H) 
      & H != john ) )). 
\end{verbatim}

Especificaci'on t'ipica en el lenguaje \textit{TPTP-FOF}. La sintaxis contiene todos los elementos de la l'ogica de primer 'orden. \textbf{![X]} es el cuantificador universal sobre la variable \textit{X}; \textbf{?[X]} es el cuantificador existencial sobre la variable \textit{X}. \textit{$human()$}, \textit{$grade()$}, \textit{$created\_equal()$} son predicados; las variables empiezan con una letra mayuscula y las constantes con minuscula. Los operadores disponibles son $\sim$, $|$, $\&$, $=>$, $<=>$, $=$ y $!=$ y corresponden con la negaci'on, la disyunci'on, la conjunci'on, la implicaci'on, la doble implicaci'on, la igualdad y la desigualdad.

En el ejemplo anterior la 'ultima f'ormula es de tipo \textit{conjecture}. 'Esto indica al demostrador que es la f'ormula que se quiere probar. Las otras f'ormulas, de tipo \textit{axiom} se interpretan como axiomas.

%á